% LaTeX template, xeCJK usepackage and Unicode text need XeLaTeX to compile.
% MiKTeX package can be downloaded at miktex.org, WinEdt can be downloaded at http://www.winedt.com/.
% WinEdt 6 and higher provide XeLaTeX and Unicode support.

% LaTeX template version 1.0.4, last revised on 2014-11-17.

\documentclass[10pt]{article}



% *************************** installed packages *************************
% AMS packages
\usepackage{amsfonts}   % TeX fonts from the American Mathematical Society.
\usepackage{amsmath}    % AMS mathematical facilities for LaTeX.
\usepackage{amssymb}    % AMS symbols
\usepackage{amsthm}     % Provide proclamations environment.

% graphics packages
\usepackage{graphics}   % Standard LaTeX graphics.
\usepackage{tikz}       % TikZ and PGF package for graphics
\usetikzlibrary{matrix} % matrix library of TikZ package
\usetikzlibrary{trees}  % trees library of TikZ package

% support for foreign languages, esp. Chinese.
\usepackage{xeCJK}      % Support for CJK (Chinese, Japanese, Korean) documents in XeLaTeX.
\setCJKmainfont{SimSun} % MUST appear when xeCJK is loaded.
%\setCJKmainfont{DFKai-SB}          % 设置正文罗马族的CKJ字体,影响 \rmfamily 和 \textrm 的字体。此处设为“标楷体”。
%\setCJKmainfont{SimSun}            % 设置正文罗马族的CKJ字体,影响 \rmfamily 和 \textrm 的字体。此处设为“宋体”。
%\setCJKmonofont{MingLiU}           % 设置正文等宽族的CJK字体,影响 \ttfamily 和 \texttt 的字体。此处设为“细明体”。
%\renewcommand\abstractname{摘要}   % 重定义摘要名:abstract->摘要。
%\renewcommand\appendixname{附录}   % 重定义附录名:appendix->附录。
%\renewcommand\bibname{参考文献}    % 重定义参考文献名:bibliography->参考文献。
%\renewcommand\contentsname{目录}   % 重定义目录名:contents->目录。
%\renewcommand\refname{参考文献}    % 重定义参考文献名:references->参考文献。

% miscellaneous packages
\usepackage[toc, page]{appendix}   % Extra control of appendices.
\usepackage{caption}    % Customising captions in floating environments.
%\captionsetup[figure]{labelformat=empty} % redefines the caption setup of the figures environment in the beamer %class.
\usepackage{clrscode}   % Typesets pseudocode as in Introduction to Algorithms.
\usepackage{epsfig}
\usepackage{eurosym}    % Metafont and macros for Euro sign.
\usepackage{float}      % Improved interface for floating objects.
\usepackage{fontspec}   % Advanced font selection in XeLaTeX and LuaLaTeX.
\usepackage{indentfirst}% Leave no indent for a paragraph after a sectional heading.
\usepackage{xcolor}     % Driver-independent color extensions for LaTeX and pdfLaTeX.

% must-be-the-last packages
\usepackage[driverfallback=hypertex, pagebackref]{hyperref}   % Extensive support for hypertext in LaTeX; MUST be on the last \usepackage line in the preamble. [pagebackref] for page referencing; [backref] for section referencing.
% ********************** end of installed packages ***********************



% ************************** fullpage.sty ********************************
% This is FULLPAGE.STY by H.Partl, Version 2 as of 15 Dec 1988.
% Document Style Option to fill the paper just like Plain TeX.
\typeout{Style Option FULLPAGE Version 2 as of 15 Dec 1988}

\topmargin 0pt \advance \topmargin by -\headheight \advance
\topmargin by -\headsep

\textheight 8.9in

\oddsidemargin 0pt \evensidemargin \oddsidemargin \marginparwidth
0.5in

\textwidth 6.5in
% For users of A4 paper: The above values are suited for American 8.5x11in
% paper. If your output driver performs a conversion for A4 paper, keep
% those values. If your output driver conforms to the TeX standard (1in/1in),
% then you should add the following commands to center the text on A4 paper:

% \advance\hoffset by -3mm  % A4 is narrower.
% \advance\voffset by  8mm  % A4 is taller.
% ************************ end of fullpage.sty ***************************



% ************** Proclamations (theorem-like structures) *****************
% [section] option provides numbering within a section.
\theoremstyle{plain}
\newtheorem{theorem}{Theorem}[section]
\newtheorem{lemma}{Lemma}[section]
\newtheorem{definition}{Definition}[section]
\newtheorem{prop}{Proposition}[section]
\newtheorem{corollary}{Corollary}[section]
\newtheorem{remark}{Remark}
\newtheorem{example}{Example}
\numberwithin{equation}{section}
\numberwithin{table}{section}
% ************************************************************************



% ************* Solutions use a modified proof environment ***************
%\newenvironment{solution}
%  {\begin{proof}[Solution]}
%  {\end{proof}}
% ************************************************************************



% ************* Frequently used commands as shorthand ********************
\newcommand{\norm}{|\!|}


\newcommand{\s}{\sigma}
\newcommand{\om}{\omega}
\newcommand{\prt}[1]{\left( #1 \right)}  % parenthese
\newcommand{\de}{\delta}
\newcommand{\pa}{\partial}
\newcommand{\E}{\mathbb{E}}

% ************************************************************************



\begin{document}

\title{Note on Volatility Surface}
\author{S.X}
\date{Version 1.1, 2018-11-26.}

\maketitle

\begin{abstract}
The note documents methodologies of volatility surface modeling. The contents are mainly extracted
from lecture notes of Tankov\cite{Tankov} and Touzi\cite{Touzi} from \'Ecole Polytechnique and the book of Derman\cite{Derman}. 
The credits go to these three authors.
\end{abstract}

\tableofcontents

\newpage

\section{Black-Scholes models and implied volatility}\label{sect_BS}

In this section, we review the elements of Black-Scholes model.

\subsection{Black-Scholes model}

The main hypotheses of Black-Scholes model is 
\begin{enumerate}
    \item No transaction cost.
    \item No restrictions on transaction size.
    \item The market is arbitrage-free.
    \item The underlying stock price follows geometric Brownian motion under the historical measure $P$
    \begin{eqnarray}
          d S_t = \mu S_t dt+\s S_t d W_t
    \label{GBM}
    \end{eqnarray}
    where $W_t$ is a standard Brownian motion, $\mu$ is the drfit and $\s$ the volatility, $\mu$和$\s$ are both constants. 
    \item Money deposited in the bank is risk free and earns a constant interest rate $r$.    
\end{enumerate}

\subsection{Self-financing portfolio}

\begin{definition}
A self-financing strategy is a strategy where where no funds are added
to or withdrawn from the portfolio after the initial date.
\end{definition}

\begin{prop}
    Let $V_t$ be the value of a self-financing portfolio at time $t$ which is consisted of stock and bonds. $\delta_t$ denotes the number of stocks in the portfolio then
    the dynamics of $V_t$ follows
    \[
        dV_t =  \prt{V_t - \de_t S_t}rdt + \de_tdS_t   
    \]
\end{prop}
\begin{proof}
    $B_t$ denotes the value of bonds in the portfolio at time $t$, then 
    \[
        V_t = \de_t S_t + B_t
    \]

    At discrete times $\{t_i\}_{i=1}^N$, by condition of self-financing, one can write
    \begin{eqnarray*}
        && B_{t_i-} + \de_{t_i-} S_{t_i} =    B_{t_i} + \de_{t_i} S_{t_i}  \\
        &\Rightarrow&  e^{r(t_i-t_{i-1})}B_{t_{i-1}} + \de_{t_i-} S_{t_i} =  B_{t_i} + \de_{t_i} S_{t_i} 
    \end{eqnarray*}

    Therefore
    \begin{eqnarray*}
         V_{t_i} - V_{t_{i-1}} &=& (e^{r(t_i-t_{i-1})}-1)B_{t_{i-1}}   + \de_{t_{i-1}}(S_{t_i}-S_{t_{i-1}}) \\
                               &=&  (e^{r(t_i-t_{i-1})}-1)(V_{t_{i-1}} - \de_{t_{i-1}}S_{t_{i-1}}) + \de_{t_{i-1}}(S_{t_i}-S_{t_{i-1}})
    \end{eqnarray*}

    Pass $\delta t \rightarrow 0$, finally it is obtained
    \[
        dV_t =  \prt{V_t - \de_t S_t}rdt + \de_tdS_t   
    \]

\end{proof}

Suppose $V_t=C(t, S_t)$ where $C(t,S_t)$ is a regulized function. Then $dV_t=dC(t,S_t)$ and by It$\hat{o}$ lemma
\begin{eqnarray*}
    d C(t,S_t) &=& \frac{\pa C}{\pa t} d t + \frac{\pa V}{\pa S} d S_t + \frac{\pa^2 V}{\pa S^2}<dS_t,dS_t>  \\
            &=&\s \frac{\pa C}{\pa S} d W_t + (\frac{\pa C}{\pa t} +\m S_t \frac{\pa C}{\pa S}+\frac{1}{2}\s^2 S_t^2\frac{\pa^2 C}{\pa S^2})dt 
\end{eqnarray*}


Because of $dV_t=dC(t,S_t)$ we necessarily have 
\begin{eqnarray}
    \de_t &=& \frac{\pa C(t,S_t)}{\pa S} \nonumber \\
    rC(t,S) &=& \frac{\pa C}{\pa t} +rS\frac{\pa C}{\pa S} + \frac{1}{2} \s^2 S^2 \frac{\pa^2 C}{\pa S^2} \label{BS_PDE}
\end{eqnarray}


From the above, we can conclude that if function $C(t,S_t)$ satisfies equation(\ref{BS_PDE}), the portfolio with $\de_t=\frac{\pa C(t,S_t)}{\pa S}$ and $B_t=C(t,S_t)-\de_t S_t$ is self-financing.
By no-arbitrage principle, this implis that the price of an option with payoff $h(S_T)$ is the solution of equation(\ref{BS_PDE}) with the boundary condition $C(T,S)=h(S_T)$.


\subsection{Risk-neutral pricing}

By assumption of Black-Scholes model, under the historical measure $P$, 
\[
    d S_t = \mu S_t dt+\s S_t d W_t    
\]

Let $Q$ be an equivalent probability measure of $P$ where 
\[
    \left.\frac{dQ}{dP}\right|_{{\cal F}_t} = 
    \exp\prt{-\frac{\mu - r}{\s} W_t - \frac{(\mu-r)^2}{2\s^2}t}
\]

By Girsanov theorem $\widehat{W}_t := W_t + \frac{\mu - r}{\s} t$ is a Brownian motion under probability measure $Q$.
and 
\[
    d S_t = \mu S_t dt+\s S_t d \widehat{W}_t    
\]

Let $\tilde{C}(t, S_t)=e^{-r(T-t)} \E^Q [h(S_T) \vert S_t ] $, then for $s<t$
\begin{eqnarray*}
    \E^Q [e^{-rt} \tilde{C}(t,S_t) \vert {\cal{F}}_s ] &=&  \E^Q [e^{-rT}  \E^Q [h(S_T) \vert S_t ] \vert {\cal{F}}_s ]  \\
    &=& e^{-rT} \E^Q [\E^Q [h(S_T) \vert S_t ] \vert {\cal{F}}_s ] \\
    &=& e^{-rT} \E^Q [h(S_T) \vert {\cal{F}}_s ]  \qquad  \prt{\textrm{by tower property}}\\
    &=& e^{-rs} e^{-r(T-s)}\E^Q [h(S_T) \vert {\cal{F}}_s ] \\
    &=& e^{-rs}\tilde C(s,S_s)
\end{eqnarray*}

therefore $e^{-rt} \tilde{C}(t,S_t)$ is a Q-martingale.

Another non-rigorous proof can be done by showing
\[
    e^{-rt} \tilde{C}(t,S_t) = e^{-rT}  \E^Q [h(S_T) \vert S_t ]  = \E^Q [\frac{h(S_T)}{e^{rT}} \vert S_t ]
\]
because the num\'eraire under probability measure $Q$ is money market account(i.e. $e^rt$), therefore $e^{-rt} \tilde{C}(t,S_t)$ is a Q-martingale.

Besides by It$\hat{o}$ we have 
\[
    d(e^{-rt}\tilde C(t,S_t)) = e^{-rt} \prt{-r \tilde C+\frac{\pa \tilde C}{\pa t}+rS\frac{\pa \tilde C}{\pa S}+\frac{1}{2}\s^2 S^2 \frac{\pa^2 \tilde C}{\pa S^2}}dt + e^{-rt} \s S\frac{\pa \tilde C}{\pa S} d \widehat{W}_t
\]

So 
\[
    e^{rt}d\prt{e^{-rt}\tilde C(t,S_t)} =  \prt{-r \tilde C+\frac{\pa \tilde C}{\pa t}+rS\frac{\pa \tilde C}{\pa S}+\frac{1}{2}\s^2 S^2 \frac{\pa^2 \tilde C}{\pa S^2}}dt + \s S\frac{\pa \tilde C}{\pa S} d \widehat{W}_t   
\]

Because $e^{-rt} \tilde{C}(t,S_t)$ is a $Q$-martingale then its drift term is zero, which further shows that $\tilde C(t,S_t)$ is a solution of equation(\ref{BS_PDE}). Because of the uniqueness of the solution we can deduce that 
the price of the option with payoff $h(S_T)$, which is a solution of equation(\ref{BS_PDE}),  in the Black-Scholes model is 
\[
    C(t,S_t) = e^{-r(T-t)}\E^Q(h(S_t e^{r(T-t)-\frac{1}{2}\s^2 (T-t)+\s \widehat{W}_{T-t}})|{\cal{F}}_t)    
\]


The above is a special case of fundamental theorem of asset pricing, which says that a market is arbitrage-free if and only if there exists a probability measure $Q$(i.e. risk neutral measure), equivalent to historical measure $P$ where the instrument price 
is the discounted expected value of future payoff.  


\subsection{Implied volatility}
\subsubsection{Option delta}
Use $I$ to denote the implied volatility of option $C(t,S_t)$, by definition we can write
\[
    C(t,S_t) = C^{BS}(t, S_t, I)    
\]

If $I$ does not depend on $S_t$ but only on $K$(Derman calls this \emph{sticky strike}), then we have
\[
   \frac{\pa C(t,S_t)}{\pa S}  =  \frac{\pa C^{BS}(t,S_t)}{\pa S}    
\]
The option delta equals to delta from the Black-Scholes model. However in reality this rarely happens and we in fact
must write
\[
    \frac{\pa C(t,S_t)}{\pa S}  =  \frac{\pa C^{BS}(t,S_t, I)}{\pa S}  + \frac{\pa C^{BS}(t,S_t,I)}{\pa I}\frac{\pa I}{\pa S} 
\]

To handle the second term in above equation, practitioners usually propose models of smile evolutions under certain 
assumptions. For example, Derman proposed regimes of \emph{sticky delta} where $I$ depends on $K/S$ but not on them separately. In that
case set $I=I(K/S)$ we have 
\[
    \frac{\pa C(t,S_t)}{\pa S}  =  \frac{\pa C^{BS}(t,S_t, I)}{\pa S} + \frac{\pa C^{BS}(t,S_t,I)}{\pa I} \frac{K}{S^2} \frac{\pa I^{'}}{\pa S} 
\]

For details please go to see Derman's book\cite{Derman}.



\subsubsection{Compared with historical volatility}
In the Black-Scholes model, the implied volatility of all options on the same
underlying must be the same and equal to the historical volatility (standard
deviation of annualized returns) of the underlying. However, when it is computed
from market-quoted option prices, one observes that
\begin{itemize}
    \item The implied volatility is always greater than the historical volatility of the
    underlying.
    \item The implied volatility of different options on the same underlying depend
    on their strikes and maturity dates.
\end{itemize}


\newpage 
\section{Delta hedging strategies}
Realized volatility is the amount of noise in the stock price, it is the coefficient
of the Wiener process in the stock returns model, it is the amount of randomness that actually transpires. Implied volatility is how the
market is pricing the option currently. Since the market does not have
perfect knowledge about the future these two numbers can and will be
different. \footnote{This section is mainly extracted from Derman \cite{Derman} and Willmot \cite{Wilmott}}

\subsection{Hedged option position}

We set up the problem in the following way
\begin{enumerate}
    \item Assume we are in the world of Black-Scholes model, where all its assumptions and valuation are valid.
    \item Assume the option
    is hedged at discrete points in time, $t_0, t_1,...t_n$, such that $t_i-t_{i-1}=\de t$ and
    with tn representing the expiration time of the option.
    \item Use $C_i = C(t_i, S_i)$ to denote the market price of the option at time $t_i$ when the
    stock price is $S_i$; use $\Delta_i=\Delta(t_i, S_i)$ to denote the number of shares of
    the stock S that we short at the start of each period i.
    \item Any cash received is
    invested at the riskless rate $r$, and any cash borrowed is funded at the same
    rate.
\end{enumerate}

We begin by set up a portfolio $V_0$ by holding the option worth $C_0$. At first time step we hedge it with $\Delta_0$ of the stock. The composition of the portfolio at $t=0$ is shown in table(\ref{table_hedging_ptf_0}). 

\begin{table}[H]
    \centering
    \begin{tabular}{|rr|}
    \hline Component & Value\\
    \hline Option&$C_{0}$\\
        Stock& $-\Delta_0 S_{0}$\\
        Cash&$\Delta_0 S_{0}$\\
    \hline
    \end{tabular}\caption{Portfolio composition and values at $t=t_0$}
    \label{table_hedging_ptf_0}
\end{table}

At the second period the composition updates, and one needs to rebalance the delta hedging position, as shown in table(\ref{table_hedging_ptf_1}) 
\begin{table}[H]
    \centering
    \begin{tabular}{|rrr|}
    \hline Component & Value before rebalance & Value after rebalance \\
    \hline Option  &   $C_{1}$                & $C_{1}$  \\ 
        Stock      & $-\Delta_0 S_{1}$        & $-\Delta_1 S_{1}$\\
        Cash      & $\Delta_0 S_0e^{r\de t}$ & $\Delta_0 S_0e^{r\de t}+(\Delta_1-\Delta_0)S_1$ \\
    \hline
    \end{tabular}\caption{Portfolio composition and values at $t=t_1$}
    \label{table_hedging_ptf_1}
\end{table}

At maturity the total value of the portfolio becomes
\[
    V_n = C_n -\Delta_n S_n +\Delta_0 S_0 e^{nr\de t}+(\Delta_1-\Delta_0)S_1 e^{(n-1)r\de t}+(\Delta_2-\Delta_1)S_2 e^{(n-2)r\de t}+ \cdots + (\Delta_n-\Delta_{n-1})S_n
\]

Let $n \rightarrow \infty$ with $n \de t = t_n-t_0 \equiv T$, we have 
\[
    V_T = C_T -\Delta_TS_T +\Delta_0S_0 e^{rT} + \int_0^T  e^{r(T-t)}S_t d\Delta_t   
\]

In the idealized Black-Scholes world, the option is perfectly hedged at every instant,
and therefore the final P$\&$L is independent of the stock price path. Because
the instantaneously hedged option is riskless, the hedging strategy replicates
a riskless bond and therefore, by the law of one price, must have the same
final value. Therefore
\[
    V_T = C_0e^{rT} = C_T -\Delta_TS_T +\Delta_0S_0 e^{rT} + \int_0^T  e^{r(T-t)}S_t d\Delta_t        
\]


\subsection{Hedging with realized volatility}

We now use $C(t,S_t; \s)$ to denote the Black-Scholes price with volatility $\s$. 

\begin{theorem}
    By delta hedging with realized volatility, the present value of P$\&$L of a hedged option position is 
    \[
        C(S_0,t_0;\s^r)-C(S_0,t_0;\s^i)    
    \]
    where $\s^r$ is the realized volatility and $\s^i$ the implied volatility.

    \label{theorem_hedging_real_vol}
\end{theorem}


\begin{proof}
    We set $C_t^r = C(t, S_t;\s^r), C_t^i = C(t, S;\s^i)$ and use $\Delta^r$ to denote the delta from BSM formula using realized volatility while $\Delta^i$
    using the implied volatility. 
    
    For simplicity, we ignore the $t$ substitutes in this section.

    Initially we buy the option at its implied volatility and hedged at known realized volatility, the hedged portfolio
    at any time $t$ is given by
    \[
        \pi = C^i - \Delta^r S    
    \]

    After $\de t$ the mark to market profit becomes
    \begin{eqnarray}
        d\pi = dC^i - \Delta^r dS - (C^i -\Delta^r S)rdt  
        \label{hedging_dynamic_real_vol}     
    \end{eqnarray}
    where the first term is the increase in the value of the long position in the option, the
    second is the decrease in the value of the short position in the stock and the last term, $(C_t^i -\Delta_t^r S_t)rdt$, represents the interest on the cost
    of borrowing an amount $(C^i -\Delta^r S)$ used to set up the initial hedge portfolio.

    Besides if we value an option at $\s^r$ and hedged at $\s^r$, the hedging strategy becomes riskless one that leads to BSM equation, With the riskless
    hedging strategy, the increase in value of the hedge portfolio should be no different from the interest earned on the position at the riskless rate, meaning 
    \begin{eqnarray}
        0 = dC^r - \Delta^r dS - (C^r -\Delta^r S)rdt 
        \label{hedging_dynamic_riskless}
    \end{eqnarray}
    
    By rearranging the teams in equation (\ref{hedging_dynamic_riskless}) we obtain 
    \begin{eqnarray}
        \Delta^r S rdt  - \Delta^r dS = - dC^r + C^r rdt    
        \label{hedging_dynamic_real_vol_relation}
    \end{eqnarray} 
    
    Substituting equation (\ref{hedging_dynamic_real_vol_relation}) in equation (\ref{hedging_dynamic_real_vol}) we arrive at
    \[
        d\pi = dC^i - dC^r - (C^i -C^r)rdt   = e^{rt} d\prt{e^{-rt}(C^i - C^r)}    
    \]

    The present value of this profit can be obtained by discounting
    \[
        e^{-r(t-t_0)}e^{rt} d\prt{e^{-rt}(C^i - C^r)}  = e^{rt_0} d\prt{e^{-rt}(C^i - C^r)} 
    \] 

    The entire profit of the hedging strategy is 
    \[
        e^{rt_0} \int_{t_0}^T d\prt{e^{-rt}(C^i - C^r)} = e^{rt_0} \prt{e^{-rt}(C^i - C^r)} \left. \right|_{t_0}^T
    \]

    At expiration, when t = T, the value of the option is simply its intrinsic value, where $C^i_T=C^r_T$, therefore the profit is then 
    \[
        e^{rt_0}\prt{e^{-rt_0} (C^i_{t_0} - C^r_{t_0})} = C^r_{t_0} - C^i_{t_0}  
    \] 

\end{proof}

Provided we know the future realized
volatility and provided that we can hedge continuously, the final P$\&$L
at the expiration of the option is known and deterministic and is equal to
the difference between the value of the option based on realized volatility
and the value of the option based on implied volatility.



\subsection{Hedging with implied volatility}
By hedging with implied volatility we are balancing
the random fluctuations in the mark-to-market option value with the
fluctuations in the stock price. The evolution of the portfolio value is
then ‘deterministic’ as we shall see.


\begin{theorem}
    By delta hedging with implied volatility, the present value of P$\&$L of a hedged option position is 
    \[
        \frac{1}{2}\prt{(\s^i)^2 - (\s^r)^2 }  \int_{t_0}^T e^{-r(t-t_0)}  S^2_t \Gamma^i dt  
    \]
    \label{theorem_hedging_imp_vol}
\end{theorem}

\begin{proof}
    By It$\hat{o}$ lemma, we have 
    \[
        dC^i =  \Theta^i dt+ \Delta^i dS_t + \frac{1}{2}(\s^r)^2 S^2_t \Gamma^i dt    
    \]

    then the instantaneous profit of the hedged position is
    \begin{eqnarray}
        d\pi_t &=& dC^i - \Delta^i dS - (C^i -\Delta^i S)rdt  \nonumber \\
        &=&  \Theta^i dt+ \Delta^i dS_t + \frac{1}{2}(\s^r)^2 S^2_t \Gamma^i dt  - \Delta^i dS - (C^i -\Delta^i S)rdt \nonumber \\
        &=&  \prt{ \Theta^i + \Delta^i rS + \frac{1}{2}(\s^r)^2 S^2_t \Gamma^i -  rC^i }dt 
        \label{hedging_dynamic_imp_vol_raw}
    \end{eqnarray}

    Because of Black-Scholes PDE
    \[
        \Theta^i + \Delta^i r S_t + \frac{1}{2}(\s^i)^2 S^2_t \Gamma^i - rC^i = 0
    \]
    equation(\ref{hedging_dynamic_imp_vol_raw}) becomes
    \begin{eqnarray*}
        d\pi_t &=& \prt{ \Theta^i + \Delta^i rS + \frac{1}{2}(\s^r)^2 S^2_t \Gamma^i -  rC^i }dt \\
            &=& \frac{1}{2}\prt{(\s^i)^2 - (\s^r)^2 } S^2_t \Gamma^i dt
    \end{eqnarray*}

    Integrate the present value of all of these profits over the life of the
    option to get a total profit of
    \[
        \frac{1}{2}\prt{(\s^i)^2 - (\s^r)^2 }  \int_{t_0}^T e^{-r(t-t_0)}  S^2_t \Gamma^i dt
    \]
    

\end{proof}



\subsection{Hedging with arbitrary constant volatility}

\begin{theorem}
    By delta hedging with constant volatility $\s^h$, the present value of P$\&$L of a hedged option position is 
    \[
        C(S_0,t_0;\s^h)-C(S_0,t_0;\s^i) + \int_{t_0}^T e^{-r(t-t_0)}((\s^r)^2-(\s^h)^2)\frac{S_t^2}{2}\frac{\pa^2 C}{\pa S^2}(S_t,t;\s^h) dt    
    \]
\end{theorem}

\begin{proof}
    Similar with previous proof, the instantaneous profit of the hedged position at time $t$ is
    \begin{eqnarray}
        d\pi &=& dC^i - \Delta^h dS - (C^i -\Delta^h S)rdt  \nonumber \\
        &=& \prt{dC^i - dC^h - r(C^i - C^h)dt} + \prt{dC^h - \Delta^h dS  - r(C^h-\Delta^h dS)dt}  
        \label{hedging_dynamic_const_vol_raw}
    \end{eqnarray}

    From proofs in Theorem(\ref{theorem_hedging_real_vol}) , we can see that the present values of first term in equation(\ref{hedging_dynamic_const_vol_raw}) equals
    \[
        C^h_{t_0} - C^i_{t_0}    
    \]

    From proofs in Theorem(\ref{theorem_hedging_imp_vol}) , the present value of second item in equation(\ref{hedging_dynamic_const_vol_raw}) arrives
    \[
        \frac{1}{2} \prt{(\s^r)^2 - (\s^h)^2 }  \int_{t_0}^T e^{-r(t-t_0)}  S^2_t \Gamma^h dt  
    \]
    
    Therefore the present value of total P$\&$L is 
    \[
        C(S_0,t_0;\s^h)-C(S_0,t_0;\s^i) + \frac{1}{2} \prt{(\s^r)^2-(\s^h)^2}\int_{t_0}^T e^{-r(t-t_0)}\frac{S_t^2}{2}\frac{\pa^2 C}{\pa S^2}(S_t,t;\s^h) dt   
    \]

\end{proof}


\newpage 
\section{Local volatility models}

In section(\ref{sect_BS}) we see that the Black-Scholes model with constant volatility
cannot reproduce all the option prices observed in the market for a given underlying
because their implied volatility varies with strike and maturity.

To take
into account the market implied volatility smile while staying within a Markovian
and complete model (one risk factor), a natural solution is to model the
volatility as a deterministic function of time and the value of the underlying
\begin{eqnarray}
  \frac{dS_t}{S_t} = rdt + \s(t, S_t)dW_t  
  \label{local_vol_sde}
\end{eqnarray}
where $r$ is the interest rate, assumed to be constant, and $W_t$ is the Brownian motion
under the risk-neutral measure $Q$. The equation(\ref{local_vol_sde}) defines a \emph{local volatility
model}.


By the same procedure of building self-financing portfolio, shown in section(\ref{sect_BS}), we see that the price of an option with payoff $h(S_T)$ at date date $T$ is given by 
\[
    C(t,S_t) = \E^Q [ e^{-r(T-t)}h(S_T) \vert S_t  ]    
\]
and is characterized by the partial differential equation
\[
    rC(t,S) = \frac{\pa C}{\pa t} +rS\frac{\pa C}{\pa S} + \frac{1}{2} \s(t,S)^2 S^2 \frac{\pa^2 C}{\pa S^2}, \quad C(T,S)=h(S)    
\]

The pricing equation has the same form
as in the Black-Scholes model, but one can no longer deduce an explicit pricing
formula, because the volatility is now a function of the underlying.


Practicioners have adopted a \emph{model calibration} approach which allows
to use all observed prices of quoted options as an input for their pricing and
hedging activities. The parameters
obtained by calibration are of course different from those which would be
obtained by historical estimation. But this does not imply any problem related
to the presence of arbitrage opportunities.


The model calibration approach is adopted in view of the fact that financial
markets do not obey to any fundamental law except the simplest no-dominance
or the slightly stronger no-arbitrage. There is no universally
accurate model in finance, and any proposed model is wrong. Therefore,
practitioners primarily base their strategies on comparison between assets, this
is exactly what calibration does.

\subsection{Dupire's formula}

\begin{theorem}
If $S_t$ follows
\[
    \frac{dS_t}{S_t} = rdt + \s(t, S_t)dW_t, S_{t_0} = S_0      
\]
\end{theorem}
and suppose that
\begin{enumerate}
    \item $S_t$ is square integrable, i.e. $\E [ \int_{t_0}^T S_t^2 dt ] < \infty$, $\forall T$.
    \item For $\forall t > t_0$, $S_t$ has a density $p(t, x)$ which is continue on $(t_0, \infty) \times (0, \infty)$.
    \item Diffusion coeffient $\s(t, x)$ is continue on $(t_0, \infty) \times (0, \infty)$.
\end{enumerate}

Then the value of the call option $C(T, K)=e^{-r(T-t)}\E [(S_T - K)^{+}]$ satisfies Dupire's equation
\[
    \frac{\pa C}{\pa T} = \frac{\s(T,K)^2 K^2}{2} \frac{\pa^2 C}{\pa K^2} - rK\frac{\pa C}{\pa K} ,  \quad (T,K) \in [t_0, \infty) \times [0, \infty) 
\] 
with the initial condition $C(t_0, K) = (S_0 - K)^+$

\begin{proof}
 Please see section (10.3) in Touzi \cite{Touzi}. 
\end{proof}

The value of the Dupire equation is that it tells you how to find a unique
local volatility function $\s(T, K)$ from the market prices of standard options.
Given the $\s(T, K)$ for all $T$ and $K$, one can then construct an implied tree
that incorporates these local volatilities to value exotic options and to hedge
standard options. 

This single, theoretically unique implied tree will value
all standard options in agreement with their market prices, and consistently
within a single model, rather than having to use an inconsistent BSM framework
with different underlying volatilities for each standard option.

The local volatility surface calculated from market prices can also be
useful for volatility arbitrage trading. One can calculate future local volatilities
implied from option prices and then decide if they seem reasonable. Butterfly and calendar spreads 
positions can be built for arbitrage if these future volatilities seem unreasonably low or high.
Please see Chapter 15 in Derman\cite{Derman} for a theoretical example.


\subsubsection{Calibration of local volatility models}
Modelling local volatility by Dupire's formula is criticized for: 
\begin{itemize}
    \item Market prices are not known for all strikes and all maturities. They must
    be interpolated and the final result is very sensitive to the interpolation
    method used.
    \item The only risk factor is underlying price and therefore is impossible to incorporate 
    volatility risk.
    \item Because of the need to calculate the second derivative of the option price
    function $C(T, K)$, small data errors lead to very large errors in the solution
    (ill-posed problem).
\end{itemize}

Due to these problems, in practice, Dupire's formula is not used directly
on the market prices. To avoid solving the ill-posed problem, practitioners
typically use one of two approaches to calibrate the model:
\begin{enumerate}
    \item Start by a preliminary calibration of a parametric functional form to the
    implied volatility surface (for example, a function quadratic in strike and
    exponential in time may be used). With this smooth parametric function,
    recalculate option prices for all strikes, which are then used to calculate
    the local volatility by Dupire's formula.
    \item Reformulate Dupire's equation as an optimization problem by introducing
    a penalty term to limit the oscillations of the volatility surface.
\end{enumerate}



\newpage
\section{Stochastic volatility models}
The local volatility model covered in previous chapters can be viewed as
a special case of a stochastic volatility model. The local volatility of a stock
varies with the stock price, and the stock price is itself stochastic. In local
volatility models, therefore, volatility is stochastic, but only because it is a
function of the stochastic stock price, with which it is 100$\%$ correlated. 

In the real world, implied and realized volatility tend to be correlated with the
underlying stock price or index level, but the correlation is not 100$\%$.

In this section we introduce an instantaneous correlation between the underlying price 
and volaility: 
\begin{eqnarray}
    &&\frac{d S_t}{S_t} = \mu_t dt + \s_t dW_t   \nonumber \\
    &&d \s_t = a_t dt + b_t d W^{'}_t, \quad d<W, W^{'}>=\rho dt
    \label{stoch_vol_sde}
\end{eqnarray}
where $a_t=a(t, \s_t, S_t), b_t=b(t, \s_t, S_t)$, $\rho \in (-1, 1)$. Because there are two source of risks, two underlyings are desired for hedging the risks. 

Assume there exists a liquid underlying with the price
\[
    C_t^0=C^0(t, \s_t, S_t)    
\]
where the deterministic function $C^0(t, \s_t, S_t)$ is known and $\frac{\pa C^0(t, \s, S)}{\pa \s} > 0$ for all $t,\s, S$. Besides some assumptions of assuring the
existence and regularity of the solution has to be made but not listed in details here.

Use $V_t$ to denote the value of a self-financing portfolio with $\de_t$ stocks and $\om_t$ quantity of $C^0$. By self-financing condition and I't$\hat{o}$ lemma, we have
\begin{eqnarray}
    dV_t &&= \prt{V_t - \de_tS_t - \om_tC^0_t} rdt + \de_tS_t + \om_tC^0_t \nonumber \\
    && =  \prt{V_t - \de_tS_t - \om_tC^0_t} rdt + \prt{\de_t + \om_t \frac{\pa C^0}{\pa S}}dS_t + \om_t \frac{\pa C^0}{\pa \s} d\s_t + \om_t {\cal{L}}_t C^0 dt
    \label{sto_vol_self_financing}
\end{eqnarray}
where 
\[
    {\cal{L}}_t = \frac{\pa}{\pa t} + \frac{1}{2} S_t^2 \s_t^2 \frac{\pa^2}{\pa S^2} + \frac{1}{2} b_t^2 \s_t^2 \frac{\pa^2}{\pa \s^2} + S_t\s_t b_t\rho \frac{\pa^2}{\pa S \pa \s}
\]

Suppose this self-financing portfolio replicates a deterministic payoff function $C: V_t =C(t, \s_t, S_t)$. By It$\hat{o}$ lemma, it is obtained that
\[
    dV_t = {\cal{L}}_t C dt + \frac{\pa C}{ \pa S}dS_t + \frac{\pa C}{ \pa \s}d\s_t 
\]

To equalize the above dynamics and equation(\ref{sto_vol_self_financing}), we arrive 
\begin{eqnarray*}
    && \om_t = \frac{\pa C / \pa \s}{\pa C^0 / \pa \s}   \\
    && \s_t = \frac{\pa C}{\pa S} - \om_t \frac{\pa C^0}{\pa S}\\
    && {\cal{L}}_tC - rC_t + rS_t \frac{\pa C}{\pa S} = \frac{\pa C}{\pa \s}  \frac{{\cal{L}}_t C^0 - rC^0 + rS_t \frac{\pa C^0}{\pa S}}{\pa C^0 / \pa \s}
\end{eqnarray*}

Set $\lambda = - \frac{{\cal{L}}_t C^0 - rC^0 + rS_t \frac{\pa C^0}{\pa S}}{\pa C^0 / \pa \s}$, note that $\lambda$ does not depend on values of option that we are going to hedge, but 
only on values of the stock that has been chosen initially. By replication we arrive at the equation that the price of the option with payoff $h(S_T)$ satisfies
\begin{eqnarray}
    {\cal{L}}_tC - rC_t + rS_t \frac{\pa C}{\pa S} + \lambda(t, \s, S) \frac{\pa C}{\pa \s}= 0, \quad C(T, \s, S) = h(S)    
    \label{stoch_vol_bs_sde}
\end{eqnarray} 
where 
\[
    {\cal{L}}_t = \frac{\pa}{\pa t} + \frac{1}{2} S_t^2 \s_t^2 \frac{\pa^2}{\pa S^2} + \frac{1}{2} b_t^2 \s_t^2 \frac{\pa^2}{\pa \s^2} + S_t\s_t b_t\rho \frac{\pa^2}{\pa S \pa \s}   
\]

This is the generalization of the Black-Scholes equation with stochastic volatility. The hedging strategy with two types of underlyings whose hedging ratio satisfying $\om_t = \frac{\pa C / \pa \s}{\pa C^0 / \pa \s}$ and 
$\s_t = \frac{\pa C}{\pa S} - \om_t \frac{\pa C^0}{\pa S}$ is called delta-vega hedging(risk of vega is risk of volatility).


\subsection{Risk-neutral valuation} 
By It$\hat{o}$ lemma, we find $C_t$ satisfies
\[
    dC_t = {\cal{L}}_t C dt + \frac{\pa C}{ \pa S}dS_t + \frac{\pa C}{ \pa \s}d\s_t   
\]

From equation(\ref{stoch_vol_bs_sde}) we see
\[
    {\cal{L}}_t C =  rC_t - rS_t \frac{\pa C}{\pa S} - \lambda(t, \s, S) \frac{\pa C}{\pa \s}
\]

Obviously by replacement of ${\cal{L}}_t C$ we have obtained
\[
    dC_t = rC_t+   \frac{\pa C}{\pa S} \prt{dS_t - rS_t dt} + \frac{\pa C}{\pa \s} \prt{ds_t - \lambda_t dt}
\]

Therefore in risk neutral measure we can write
\begin{eqnarray*}
    && \frac{dS_t}{S_t} = rdt + \s_t d \tilde{W}_t \\
    && d\s_t = \lambda_t dt + b_t  d \tilde{W^{'}}_t
\end{eqnarray*} 
where $\tilde{W}_t, \tilde{W^{'}}_t$ are two Brownian Motion under risk neutral measure $Q$ 
with $d<\tilde{W}, \tilde{W^{'}}>=\rho dt$.

Under $Q$, the price of European option satisfies
\[
    C(t, \s, S)  = \E^Q [e^{-r(T-t)} h(S_T) \vert  \s_t=\s, S_t=S ]    
\]


\subsection{Symmetric principle of implied volatility}
As in the Black-Scholes model 
\[
    \frac{dS_t}{S_t} = rdt + \s dW_t   
\]
$e^{-rt}S_t$ is a martingale under risk neutral measure $Q$. We introduce a new measure defined by
\[
    \left.\frac{dQ^*}{dQ}\right|_{{\cal F}_t} = \frac{S_t}{e^{rt}S_0}=e^{\s W_t - \frac{1}{2}\s^2 t}
\]
then by Girsanov's theorem $W^*_t = W_t - \s t$ is a Brownian Motion under measure $Q^*$. By symmetry property $\tilde{W}=-W^*$ is an equivalent Brownian Motion
under measure $Q^*$. So we can find that 
\begin{eqnarray*}
    e^{2rt}\frac{S^2_0}{S_t} &&= e^{2rt}S_0 e^{-\s W_t - (r-\frac{\s^2}{2})t} \\
    && = S_0 e^{-\s (W^*_t + \s t) + (r+\frac{\s^2}{2})t}\\
    && = S_0  e^{-\s W^*_t +(r-\frac{\s^2}{2})t} \\
    && = S_0  e^{\s \tilde{W}_t +(r-\frac{\s^2}{2})t}
\end{eqnarray*} 
which implies that the distribution of $e^{2rt}\frac{S^2_0}{S_t}$ under measure $Q^*$ is the same with that of $S_t$ under measure 
$Q$. 

\begin{theorem}
    Use $C^{BS}(K,T,\s), P^{BS}(K,T,\s)$ denote the Black-Scholes call and put price with strike $K$, maturity $T$ and volatility $\s$. Then 
    \[
        C^{BS}(K,T,\s) = \frac{K}{s_0e^{rT}}P^{BS}\prt{\frac{(r^{rT}S_0)^2}{K},T,\s}    
    \]
\end{theorem}
\begin{proof}

    By using the result derived before the theorem, we can write 
    \begin{eqnarray*}
        C^{BS}(K, T, \s) &&= e^{-rT} \E[(S_T-K)^+] \\
        &&= e^{-rT} \frac{K}{S_0 e^{rT}} \E \left[ \frac{S_T}{e^{rT}S_0} \prt{\frac{(e^{rT} S_0)^2}{K} - \frac{(e^{rT} S_0)^2}{S_T} }^+ \right] \\
        &&= e^{-rT} \frac{K}{S_0 e^{rT}} \E^* \left[\prt{\frac{(e^{rT} S_0)^2}{K} - \frac{(e^{rT} S_0)^2}{S_T} }^+  \right] \\
        &&= e^{-rT} \frac{K}{S_0 e^{rT}} \E^* \left[\prt{\frac{(e^{rT} S_0)^2}{K} - S_0  e^{\s \tilde{W}_t +(r-\frac{\s^2}{2})t} }^+  \right] \\
        &&=\frac{K}{s_0e^{rT}}P^{BS}\prt{\frac{(r^{rT}S_0)^2}{K},T,\s}  
    \end{eqnarray*}
\end{proof}


TODO: add stoch vol case.

\subsection{The characteristic solution}
First introduce the following theorem without proof.

\begin{theorem}(Mixing Theorem, Romano and Touzi, 1997)
    Let $C(S_0,T, \s_0) $ be the call option price under
    the risk-adjusted, stochastic volatility process (\ref{stoch_vol_sde}) given today's price $S_0$ and volatility $\s_0$ of underlying. 
    Let $C^{BS}(S_0, T, \s_0)$ be the Black-Scholes price. Define 
    \begin{eqnarray*}
        && S^e_T=S_0\exp \prt{-\frac{1}{2}\int^T_0 \rho_t^2 \s_t^2 dt + \int^T_0 \rho_t  \s_t d W_t } \\
        && V^e_T=\frac{1}{T} \int_0^T (1-\rho^2) \s_t^2 dt \\
        && \left< C(S_T^e, T, \sqrt{V_T^e}) \right> = \lim_{\Delta_t \rightarrow 0} \int^{\infty}_{-\infty} \int^{\infty}_{-\infty}...C^{BS}(S_T^e, T, \sqrt{V_T^e}) \prod_{t=0}^{T-\Delta t} \exp(-\frac{1}{2} {W^{'}}^2_t )
        \frac{d W^{'}_t}{(2\pi)^{0.5}}
    \end{eqnarray*}
    then the call option price satisfies
    \[
        C(S_0,T, \s_0) = \left< C(S_T^e, T, \sqrt{V_T^e}) \right>
    \]

\end{theorem}

In words, the option value under stochastic volatility is a weighted sum or mixture of the
Black-Scholes values with an effective stock price and effective volatility. The effective
variables depend only upon the volatility process. Hence, the valuation reduces to a pricing
expectation over the risk-adjusted volatility process alone. 

\subsubsection{Zero correlation case}
In general $<S^e_T, S^e_T> = S_0$, when $\rho=0$, then $S^e_T=S_0$. 

Define the integrated variance $V_T = \int_0^T \s^2_t dt$. In fact $\frac{V_T}{T} = \frac{1}{T}\int_0^T \s^2_t dt$ can be 
seen as the average variance along a path.

We introduce the density 
function $p(\s_t)$ of $\s_t$. From mixing theorem we have
\[
    C(T,K) = \int_0^{\infty} C^{BS}(T, K, \sqrt{V_T/T}) p(\s_t) d \s_t = \E [C^{BS}(T, K, \sqrt{V_T/T})]
\]
Hull and White(1987) established this result which will not derived in detail here. 

Set $\overline{C}^BS(T,K,V_T) :=C^{BS}(T, K, \sqrt{V_T/T})$, we can rewrite
\[
    C(T,K) =  \E [\overline{C}(T,K,V_T)]  
\]

Suppose $V_T$ is close to $\E[V_T]$, i.e, $VarV_T \ll E[V_T^2]$. 
By expanding $\overline{C}^{BS}(T,K,V_T)$ at $V_T=E[V_T]$ we have 
\[
    C(T,K) \approx \overline{C}^{BS}(T, K, E[V_T]) + \frac{1}{2} VarV_T \frac{\pa^2}{\pa V^2} \overline{C}^{BS}(T,K,E[V_T])   
\]
where
\[
    \frac{\pa^2}{\pa V^2} \overline{C}^{BS}(T,K,E[V_T])  = \frac{SN(d_1)}{4V^{3/2}}\prt{ \frac{k^2}{V} - \frac{V}{4} -1 }, k=\log \frac{K}{Se^{rT}}   
\]
Besides, when $\de \s \ll \s$,
\[
    C^{BS}(T, K, \s+\de \s) \approx C^{BS}(T, K, \s) + \de \s \frac{\pa C^{BS}(T, K, \s)}{\pa \s}
\]
then set $\s=\sqrt{E[V_T/ T]}, \de \s = I(T,K) - \sqrt{E[V_T/ T]}$
\begin{eqnarray*}
    C^{BS}(T, K, I(T,K)) && \approx C^{BS}(T, K, \sqrt{E[V_T/ T]}) + \prt{I(T,K) - \sqrt{E[V_T/ T]}} \frac{\pa C^{BS}(T, K, \sqrt{E[V_T/ T]})}{\pa \s} \\
    && = \overline{C}^{BS}(T,K,E[V_T]) + \prt{I(T,K) - \sqrt{E[V_T/ T]}} \frac{\pa C^{BS}(T, K, \sqrt{E[V_T/ T]})}{\pa \s}
\end{eqnarray*}

Assuming $C(T,K) = C^{BS}(T, K, I(T,K))$, then it can be easily obtained that 
\begin{eqnarray}
    &&\frac{1}{2} VarV_T \frac{\pa^2}{\pa V^2} \overline{C}^{BS}(T,K,E[V_T])  = \prt{I(T,K) - \sqrt{E[V_T/ T]}} \frac{\pa C^{BS}(T, K, \sqrt{E[V_T/ T]})}{\pa \s} \nonumber \\
    &\Rightarrow& I(T,K) = \sqrt{E[V_T/ T]} + \frac{1}{2}  VarV_T \frac{\frac{\pa^2}{\pa V^2} \overline{C}^{BS}(T,K,E[V_T]) }{\frac{\pa C^{BS}(T, K, \sqrt{E[V_T/ T]})}{\pa \s}} \nonumber \\
    &\Rightarrow& I(T,K) = \sqrt{E[V_T/ T]} + \frac{1}{8T^{1/2}E[V_T]^{3/2}}  VarV_T \prt{ \frac{k^2}{E[V_T]} - \frac{E[V_T]}{4} -1 }
    \label{stoch_imp_vol_sym}
\end{eqnarray}

Equation(\ref{stoch_imp_vol_sym}) shows a symmetric form of smile in log-moneyness $k$.



%where $H(t) = \int_0^t e^{-\kappa s}ds$ and $\zeta_t = \int_0^t e^{2\kappa s}\sigma_s^2 ds$ ($\sigma_{\cdot}$ is the volatility parameter in the corresponding one-factor Hull-White model).

%Therefore, calibration to caplets in order to obtain $\zeta_{t_f}$ requires solving the following equation
%\[
%%\boxed{
%\mbox{Bl}(K, F, \sigma_B\sqrt{t_f}, 1) = \mbox{Bl}\left(K+\frac{1}{\tau}, F+\frac{1}{\tau}, \left[H(T_e)-H(T_s)\right] \sqrt{\zeta_{t_f}}, 1\right)
%}
%\]
%We note $\mbox{Bl}(K, F, v, 1)$ is a monotone increasing function of $v$ with a range of $((F-K)^+, F)$. So the above calibration equation always has a solution.



\begin{appendix}


\end{appendix}
\newpage

\begin{thebibliography}{99}

\bibitem{Lewis} Alan Lewis. The mixing approach to stochastic volatility and jump models. {\it
Wilmott}, 2002.

\bibitem{Wilmott} Riaz Ahmad and Paul Wilmott. Which free lunch would you like today, sir? delta hedging, volatility arbitrage and optimal portfolios. {\it
Wilmott}, 2005.
    
\bibitem{Derman} Emanuel Derman and Michael B.Miller. The volatility smile. {\it
Weiley Financial Series}, 2016.

\bibitem{Touzi} Nizar Touzi. Calcul Stochatique en Finance. {\it
Lecture Note} {\bf}, PA Math\'ematiques Appliqu\'ees, Ecole Polytechnique, September, 2018.

\bibitem{Tankov} Peter Tankov, Surface de volatilit\'e, {\it
Lecture Note of Master Mod\'elisation Al\'eatoire \`a Paris 7}, ENSAE ParisTech, February, 2015.

\end{thebibliography}

\end{document} 