% LaTeX template, xeCJK usepackage and Unicode text need XeLaTeX to compile.
% MiKTeX package can be downloaded at miktex.org, WinEdt can be downloaded at http://www.winedt.com/.
% WinEdt 6 and higher provide XeLaTeX and Unicode support.

% LaTeX template version 1.0.4, last revised on 2014-11-17.

\documentclass[10pt]{article}



% *************************** installed packages *************************
% AMS packages
\usepackage{amsfonts}   % TeX fonts from the American Mathematical Society.
\usepackage{amsmath}    % AMS mathematical facilities for LaTeX.
\usepackage{amssymb}    % AMS symbols
\usepackage{amsthm}     % Provide proclamations environment.

% graphics packages
\usepackage{graphics}   % Standard LaTeX graphics.
\usepackage{tikz}       % TikZ and PGF package for graphics
\usetikzlibrary{matrix} % matrix library of TikZ package
\usetikzlibrary{trees}  % trees library of TikZ package

% support for foreign languages, esp. Chinese.
\usepackage{xeCJK}      % Support for CJK (Chinese, Japanese, Korean) documents in XeLaTeX.
\setCJKmainfont{SimSun} % MUST appear when xeCJK is loaded.
%\setCJKmainfont{DFKai-SB}          % 设置正文罗马族的CKJ字体,影响 \rmfamily 和 \textrm 的字体。此处设为“标楷体”。
%\setCJKmainfont{SimSun}            % 设置正文罗马族的CKJ字体,影响 \rmfamily 和 \textrm 的字体。此处设为“宋体”。
%\setCJKmonofont{MingLiU}           % 设置正文等宽族的CJK字体,影响 \ttfamily 和 \texttt 的字体。此处设为“细明体”。
%\renewcommand\abstractname{摘要}   % 重定义摘要名:abstract->摘要。
%\renewcommand\appendixname{附录}   % 重定义附录名:appendix->附录。
%\renewcommand\bibname{参考文献}    % 重定义参考文献名:bibliography->参考文献。
%\renewcommand\contentsname{目录}   % 重定义目录名:contents->目录。
%\renewcommand\refname{参考文献}    % 重定义参考文献名:references->参考文献。

% miscellaneous packages
\usepackage[toc, page]{appendix}   % Extra control of appendices.
\usepackage{caption}    % Customising captions in floating environments.
%\captionsetup[figure]{labelformat=empty} % redefines the caption setup of the figures environment in the beamer %class.
\usepackage{clrscode}   % Typesets pseudocode as in Introduction to Algorithms.
\usepackage{epsfig}
\usepackage{eurosym}    % Metafont and macros for Euro sign.
\usepackage{float}      % Improved interface for floating objects.
\usepackage{fontspec}   % Advanced font selection in XeLaTeX and LuaLaTeX.
\usepackage{indentfirst}% Leave no indent for a paragraph after a sectional heading.
\usepackage{xcolor}     % Driver-independent color extensions for LaTeX and pdfLaTeX.

% must-be-the-last packages
\usepackage[driverfallback=hypertex, pagebackref]{hyperref}   % Extensive support for hypertext in LaTeX; MUST be on the last \usepackage line in the preamble. [pagebackref] for page referencing; [backref] for section referencing.
% ********************** end of installed packages ***********************



% ************************** fullpage.sty ********************************
% This is FULLPAGE.STY by H.Partl, Version 2 as of 15 Dec 1988.
% Document Style Option to fill the paper just like Plain TeX.
\typeout{Style Option FULLPAGE Version 2 as of 15 Dec 1988}

\topmargin 0pt \advance \topmargin by -\headheight \advance
\topmargin by -\headsep

\textheight 8.9in

\oddsidemargin 0pt \evensidemargin \oddsidemargin \marginparwidth
0.5in

\textwidth 6.5in
% For users of A4 paper: The above values are suited for American 8.5x11in
% paper. If your output driver performs a conversion for A4 paper, keep
% those values. If your output driver conforms to the TeX standard (1in/1in),
% then you should add the following commands to center the text on A4 paper:

% \advance\hoffset by -3mm  % A4 is narrower.
% \advance\voffset by  8mm  % A4 is taller.
% ************************ end of fullpage.sty ***************************



% ************** Proclamations (theorem-like structures) *****************
% [section] option provides numbering within a section.
\theoremstyle{plain}
\newtheorem{theorem}{Theorem}[section]
\newtheorem{lemma}{Lemma}[section]
\newtheorem{definition}{Definition}[section]
\newtheorem{prop}{Proposition}[section]
\newtheorem{corollary}{Corollary}[section]
\newtheorem{remark}{Remark}
\newtheorem{example}{Example}
\numberwithin{equation}{section}
\numberwithin{table}{section}
% ************************************************************************



% ************* Solutions use a modified proof environment ***************
%\newenvironment{solution}
%  {\begin{proof}[Solution]}
%  {\end{proof}}
% ************************************************************************



% ************* Frequently used commands as shorthand ********************
\newcommand{\norm}{|\!|}


\newcommand{\s}{\sigma}
\newcommand{\om}{\omega}
\newcommand{\prt}[1]{\left( #1 \right)}  % parenthese
\newcommand{\de}{\delta}
\newcommand{\pa}{\partial}
\newcommand{\E}{\mathbb{E}}

% ************************************************************************



\begin{document}

\title{Note on Volatility Surface}
\author{S.X}
\date{Version 1.0, 2018-11-12.}

\maketitle

\begin{abstract}
The note documents methodologies of volatility surface modeling and calibrations.
\end{abstract}

\tableofcontents

\newpage

\section{Black-Scholes models and implied volatility}\label{sect_BS}

In this section, we review the elements of Black-Scholes model.

\subsection{Black-Scholes model}

The main hypotheses of Black-Scholes model is 
\begin{enumerate}
    \item No transaction cost.
    \item No restrictions on transaction size.
    \item The market is arbitrage-free.
    \item The underlying stock price follows geometric Brownian motion under the historical measure $P$
    \begin{eqnarray}
          d S_t = \mu S_t dt+\s S_t d W_t
    \label{GBM}
    \end{eqnarray}
    where $W_t$ is a standard Brownian motion, $\mu$ is the drfit and $\s$ the volatility, $\mu$和$\s$ are both constants. 
    \item Money deposited in the bank is risk free and earns a constant interest rate $r$.    
\end{enumerate}

\subsection{Self-financing portfolio}

\begin{definition}
A self-financing strategy is a strategy where where no funds are added
to or withdrawn from the portfolio after the initial date.
\end{definition}

\begin{prop}
    Let $V_t$ be the value of a self-financing portfolio at time $t$ which is consisted of stock and bonds. $\delta_t$ denotes the number of stck in the portfolio then
    the dynamics of $V_t$ follows
    \[
        dV_t =  \prt{V_t - \de_t S_t}rdt + \de_tdS_t   
    \]
\end{prop}
\begin{proof}
    $B_t$ denotes the value of bonds in the portfolio at time $t$. 
    \[
        V_t = \de_t S_t + B_t
    \]

    At discrete times $\{t_i\}_{i=1}^N$, by condition of self-financing, one can write
    \begin{eqnarray*}
        && B_{t_i-} + \de_{t_i-} S_{t_i} =    B_{t_i} + \de_{t_i} S_{t_i}  \\
        &\Rightarrow&  e^{r(t_i-t_{i-1})}B_{t_{i-1}} + \de_{t_i-} S_{t_i} =  B_{t_i} + \de_{t_i} S_{t_i} 
    \end{eqnarray*}

    Therefore
    \begin{eqnarray*}
         V_{t_i} - V_{t_{i-1}} &=& (e^{r(t_i-t_{i-1})}-1)B_{t_{i-1}}   + \de_{t_{i-1}}(S_{t_i}-S_{t_{i-1}}) \\
                               &=&  (e^{r(t_i-t_{i-1})}-1)(V_{t_{i-1}} - \de_{t_{i-1}}S_{t_{i-1}}) + \de_{t_{i-1}}(S_{t_i}-S_{t_{i-1}})
    \end{eqnarray*}

    Pass $\delta t \rightarrow 0$, finally it is obtained
    \[
        dV_t =  \prt{V_t - \de_t S_t}rdt + \de_tdS_t   
    \]

\end{proof}

Suppose $V_t=C(t, S_t)$ where $C(t,S_t)$ is a regulized function. Then $dV_t=fC(t,S_t)$ and by It$\hat{o}$ lemma
\begin{eqnarray*}
    d C(t,S_t) &=& \frac{\pa C}{\pa t} d t + \frac{\pa V}{\pa S} d S_t + \frac{\pa^2 V}{\pa S^2}<dS_t,dS_t>  \\
            &=&\s \frac{\pa C}{\pa S} d W_t + (\frac{\pa C}{\pa t} +\m S_t \frac{\pa C}{\pa S}+\frac{1}{2}\s^2 S_t^2\frac{\pa^2 C}{\pa S^2})dt 
\end{eqnarray*}


Because of $dV_t=dC(t,S_t)$ we necessarily have 
\begin{eqnarray}
    \d_t &=& \frac{\pa C(t,S_t)}{\pa S} \nonumber \\
    rC(t,S) &=& \frac{\pa C}{\pa t} +rS\frac{\pa C}{\pa S} + \frac{1}{2} \s^2 S^2 \frac{\pa^2 C}{\pa S^2} \label{BS_PDE}
\end{eqnarray}


From the above, we can conclude that if function $C(t,S_t)$ satisfies equation(\ref{BS_PDE}), the portfolio with $\de_t=\frac{\pa C(t,S_t)}{\pa S}$ and $B_t=C(t,S_t)-\de_t S_t$ is self-financing.
By no-arbitrage principle, this implis that the price of an option with payoff $h(S_T)$ is the solution of equation(\ref{BS_PDE}) with the boundary condition $C(T,S)=h(S_T)$.


\subsection{Risk-neutral pricing}

By assumption of Black-Scholes model, under the historical measure $P$, 
\[
    d S_t = \mu S_t dt+\s S_t d W_t    
\]

Let $Q$ be an equivalent probability measure of $P$ where 
\[
    \left.\frac{dQ}{dP}\right|_{{\cal F}_t} = 
    \exp\prt{-\frac{\mu - r}{\s} W_t - \frac{(\mu-r)^2}{2\s^2}t}
\]

By Girsanov theorem $\widehat{W}_t := W_t + \frac{\mu - r}{\s} t$ is a Brownian motion under probability measure $Q$.
and 
\[
    d S_t = \mu S_t dt+\s S_t d \widehat{W}_t    
\]

Let $\tilde{C}(t, S_t)=e^{-r(T-t)} \E^Q [h(S_T) \vert S_t ] $, then for $s<t$
\begin{eqnarray*}
    \E^Q [e^{-rt} \tilde{C}(t,S_t) \vert {\cal{F}}_s ] &=&  \E^Q [e^{-rT}  \E^Q [h(S_T) \vert S_t ] \vert {\cal{F}}_s ]  \\
    &=& e^{-rT} \E^Q [\E^Q [h(S_T) \vert S_t ] \vert {\cal{F}}_s ] \\
    &=& e^{-rT} \E^Q [h(S_T) \vert {\cal{F}}_s ]  \qquad  \prt{\textrm{by tower property}}\\
    &=& e^{-rs} e^{-r(T-s)}\E^Q [h(S_T) \vert {\cal{F}}_s ] \\
    &=& e^{-rs}\tilde C(s,S_s)
\end{eqnarray*}

therefore $e^{-rt} \tilde{C}(t,S_t)$ is a Q-martingale.

Another non-rigorous proof can be done by showing
\[
    e^{-rt} \tilde{C}(t,S_t) = e^{-rT}  \E^Q [h(S_T) \vert S_t ]  = \E^Q [\frac{h(S_T)}{e^{rT}} \vert S_t ]
\]
because the num\'eraire under probability measure $Q$ is money market account(i.e. $e^rt$), therefore $e^{-rt} \tilde{C}(t,S_t)$ is a Q-martingale.

Besides by It$\hat{o}$ we have 
\[
    d(e^{-rt}\tilde C(t,S_t)) = e^{-rt} \prt{-r \tilde C+\frac{\pa \tilde C}{\pa t}+rS\frac{\pa \tilde C}{\pa S}+\frac{1}{2}\s^2 S^2 \frac{\pa^2 \tilde C}{\pa S^2}}dt + e^{-rt} \s S\frac{\pa \tilde C}{\pa S} d \widehat{W}_t
\]

So 
\[
    e^{rt}d\prt{e^{-rt}\tilde C(t,S_t)} =  \prt{-r \tilde C+\frac{\pa \tilde C}{\pa t}+rS\frac{\pa \tilde C}{\pa S}+\frac{1}{2}\s^2 S^2 \frac{\pa^2 \tilde C}{\pa S^2}}dt + \s S\frac{\pa \tilde C}{\pa S} d \widehat{W}_t   
\]

Because $e^{-rt} \tilde{C}(t,S_t)$ is a $Q$-martingale then its drift term is zero, which further shows that $\tilde C(t,S_t)$ is a solution of equation(\ref{BS_PDE}). Because of the uniqueness of the solution we can deduce that 
the price of the option with payoff $h(S_T)$, which is a solution of equation(\ref{BS_PDE}),  in the Black-Scholes model is 
\[
    C(t,S_t) = e^{-r(T-t)}\E^Q(h(S_t e^{r(T-t)-\frac{1}{2}\s^2 (T-t)+\s \widehat{W}_{T-t}})|{\cal{F}}_t)    
\]


The above is a special case of fundamental theorem of asset pricing, which says that a market is arbitrage-free if and only if there exists a probability measure $Q$(i.e. risk neutral measure), equivalent to historical measure $P$ where the instrument price 
is the discounted expected value of future payoff.  


\subsection{Implied volatility}

In the Black-Scholes model, the implied volatility of all options on the same
underlying must be the same and equal to the historical volatility (standard
deviation of annualized returns) of the underlying. However, when I is computed
from market-quoted option prices, one observes that
\begin{itemize}
    \item The implied volatility is always greater than the historical volatility of the
    underlying.
    \item The implied volatility of different options on the same underlying depend
    on their strikes and maturity dates.
\end{itemize}


\section{Delta hedging strategies}
Realized volatility is the amount of noise in the stock price, it is the coefficient
of the Wiener process in the stock returns model, it is the amount of randomness that actually transpires. Implied volatility is how the
market is pricing the option currently. Since the market does not have
perfect knowledge about the future these two numbers can and will be
different. \footnote{This section is mainly extracted from Derman \cite{Derman} and Willmot \cite{Wilmott}}

\subsection{Hedged option position}

We set up the problem in the following way
\begin{enumerate}
    \item Assume the option
    is hedged at discrete points in time, $t_0, t_1,...t_n$, such that $t_i-t_{i-1}=\de t$ and
    with tn representing the expiration time of the option.
    \item Use $C_i = C(t_i, S_i)$ to denote the market price of the option at time $t_i$ when the
    stock price is $S_i$; use $\Delta_i=\Delta(t_i, S_i)$ to denote the number of shares of
    the stock S that we short at the start of each period i.
    \item Any cash received is
    invested at the riskless rate $r$, and any cash borrowed is funded at the same
    rate.
\end{enumerate}

We begin by set up a portfolio $V_0$ by holding the option worth $C_0$. At first time step we hedge it with $\Delta_0$ of the stock. The composition of the portfolio at $t=0$ is shown in table(\ref{table_hedging_ptf_0}). 

\begin{table}[H]
    \centering
    \begin{tabular}{|rr|}
    \hline Component & Value\\
    \hline Option&$C_{0}$\\
        Stock& $-\Delta_0 S_{0}$\\
        Cash&$\Delta_0 S_{0}$\\
    \hline
    \end{tabular}\caption{Portfolio composition and values at $t=t_0$}
    \label{table_hedging_ptf_0}
\end{table}

At the second period the composition updates, and one needs to rebalance the delta hedging position, as shown in table(\ref{table_hedging_ptf_1}) 
\begin{table}[H]
    \centering
    \begin{tabular}{|rrr|}
    \hline Component & Value before rebalance & Value after rebalance \\
    \hline Option  &   $C_{1}$                & $C_{1}$  \\ 
        Stock      & $-\Delta_0 S_{1}$        & $-\Delta_1 S_{1}$\\
        Cash      & $\Delta_0 S_0e^{r\de t}$ & $\Delta_0 S_0e^{r\de t}+(\Delta_1-\Delta_0)S_1$ \\
    \hline
    \end{tabular}\caption{Portfolio composition and values at $t=t_1$}
    \label{table_hedging_ptf_1}
\end{table}

At maturity the total value of the portfolio becomes
\[
    V_n = C_n -\Delta_n S_n +\Delta_0 S_0 e^{nr\de t}+(\Delta_1-\Delta_0)S_1 e^{(n-1)r\de t}+(\Delta_2-\Delta_1)S_2 e^{(n-2)r\de t}+ \cdots + (\Delta_n-\Delta_{n-1})S_n
\]

Let $n \rightarrow \infty$ with $n \de t = t_n-t_0 \equiv T$, we have 
\[
    V_T = C_T -\Delta_TS_T +\Delta_0S_0 e^{rT} + \int_0^T  e^{r(T-t)}S_t d\Delta_t   
\]

In the idealized Black-Scholes world, the option is perfectly hedged at every instant,
and therefore the final P$\&$L is independent of the stock price path. Because
the instantaneously hedged option is riskless, the hedging strategy replicates
a riskless bond and therefore, by the law of one price, must have the same
final value. Therefore
\[
    V_T = C_0e^{rT} = C_T -\Delta_TS_T +\Delta_0S_0 e^{rT} + \int_0^T  e^{r(T-t)}S_t d\Delta_t        
\]


\subsection{Hedging with realized volatility}

We now use $C(t,S_t; \s)$ to denote the Black-Scholes price with volatility $\s$. 

\begin{theorem}
    By delta hedging with realized volatility, the present value of P$\&$L of a hedged option position is 
    \[
        C(S_0,t_0;\s^r)-C(S_0,t_0;\s^i)    
    \]
    where $\s^r$ is the realized volatility and $\s^i$ the implied volatility.

    \label{theorem_hedging_real_vol}
\end{theorem}


\begin{proof}
    For simplicity, we use $C^r = C(S_0,t_0;\s^r), C^i = C(S_0,t_0;\s^i)$

    Initially we buy the option at its implied volatility and hedged at known realized volatility, the hedged portfolio
    at any time $t$ is given by
    \[
        \pi = C^i - \Delta^r S    
    \]

    After $\de t$ the mark to market profit becomes
    \begin{eqnarray}
        d\pi = dC^i - \Delta^r dS - (C^i -\Delta^r S)rdt  
        \label{hedging_dynamic_real_vol}     
    \end{eqnarray}
    where the first term is the increase in the value of the long position in the option, the
    second is the decrease in the value of the short position in the stock and the last term, $(C^i -\Delta^r S)rdt$, represents the interest on the cost
    of borrowing an amount $(C^i -\Delta^r S)$ used to set up the initial hedge portfolio.

    Besides if we value an option at $\s^i$ and hedged at $\s^i$, the hedging strategy becomes riskless one that leads to BSM equation, With the riskless
    hedging strategy, the increase in value of the hedge portfolio should be no different from the interest earned on the position at the riskless rate, meaning 
    \begin{eqnarray}
        0 = dC^r - \Delta^r dS - (C^r -\Delta^r S)rdt 
        \label{hedging_dynamic_riskless}
    \end{eqnarray}
    
    By rearranging the teams in equation (\ref{hedging_dynamic_riskless}) we obtain 
    \begin{eqnarray}
        \Delta^r S rdt  - \Delta^r dS = - dC^r + C^r rdt    
        \label{hedging_dynamic_real_vol_relation}
    \end{eqnarray} 
    
    Substituting equation (\ref{hedging_dynamic_real_vol_relation}) in equation (\ref{hedging_dynamic_real_vol}) we arrive at
    \[
        d\pi = dC^i - dC^r - (C^i -C^r)rdt   = e^{rt} d\prt{e^{-rt}(C^i - C^r)}    
    \]

    The present value of this profit can be obtained by discounting
    \[
        e^{-r(t-t_0)}e^{rt} d\prt{e^{-rt}(C^i - C^r)}  = e^{rt_0} d\prt{e^{-rt}(C^i - C^r)} 
    \] 

    The entire profit of the hedging strategy is 
    \[
        e^{rt_0} \int_{t_0}^T d\prt{e^{-rt}(C^i - C^r)} = e^{rt_0} \prt{e^{-rt}(C^i - C^r)} \left. \right|_{t_0}^T
    \]

    At expiration, when t = T, the value of the option is simply its intrinsic value, where $C^i_T=C^r_T$, therefore the profit is then 
    \[
        e^{rt_0}\prt{e^{-rt_0} (C^i_{t_0} - C^r_{t_0})} = C^r_{t_0} - C^i_{t_0}  
    \] 

\end{proof}

Provided we know the future realized
volatility and provided that we can hedge continuously, the final P$\&$L
at the expiration of the option is known and deterministic and is equal to
the difference between the value of the option based on realized volatility
and the value of the option based on implied volatility.



\subsection{Hedging with implied volatility}
By hedging with implied volatility we are balancing
the random fluctuations in the mark-to-market option value with the
fluctuations in the stock price. The evolution of the portfolio value is
then ‘deterministic’ as we shall see.


\begin{theorem}
    By delta hedging with implied volatility, the present value of P$\&$L of a hedged option position is 
    \[
        \frac{1}{2}\prt{(\s^i)^2 - (\s^r)^2 }  \int_{t_0}^T e^{-r(t-t_0)}  S^2_t \Gamma^i dt  
    \]
    \label{theorem_hedging_imp_vol}
\end{theorem}

\begin{proof}
    By It$\hat{o}$ lemma, we have 
    \[
        dC^i =  \Theta^i dt+ \Delta^i dS_t + \frac{1}{2}(\s^r)^2 S^2_t \Gamma^i dt    
    \]

    then the instantaneous profit of the hedged position is
    \begin{eqnarray}
        d\pi_t &=& dC^i - \Delta^i dS - (C^i -\Delta^i S)rdt  \nonumber \\
        &=&  \Theta^i dt+ \Delta^i dS_t + \frac{1}{2}(\s^r)^2 S^2_t \Gamma^i dt  - \Delta^i dS - (C^i -\Delta^i S)rdt \nonumber \\
        &=&  \prt{ \Theta^i + \Delta^i rS + \frac{1}{2}(\s^r)^2 S^2_t \Gamma^i -  rC^i }dt 
        \label{hedging_dynamic_imp_vol_raw}
    \end{eqnarray}

    Because of Black-Scholes PDE
    \[
        \Theta^i + \Delta^i r S_t + \frac{1}{2}(\s^i)^2 S^2_t \Gamma^i - rC^i = 0
    \]
    equation(\ref{hedging_dynamic_imp_vol_raw}) becomes
    \begin{eqnarray*}
        d\pi_t &=& \prt{ \Theta^i + \Delta^i rS + \frac{1}{2}(\s^r)^2 S^2_t \Gamma^i -  rC^i }dt \\
            &=& \frac{1}{2}\prt{(\s^i)^2 - (\s^r)^2 } S^2_t \Gamma^i dt
    \end{eqnarray*}

    Integrate the present value of all of these profits over the life of the
    option to get a total profit of
    \[
        \frac{1}{2}\prt{(\s^i)^2 - (\s^r)^2 }  \int_{t_0}^T e^{-r(t-t_0)}  S^2_t \Gamma^i dt
    \]
    

\end{proof}



\subsection{Hedging with arbitrary constant volatility}

\begin{theorem}
    By delta hedging with constant volatility $\s^h$, the present value of P$\&$L of a hedged option position is 
    \[
        C(S_0,t_0;\s^h)-C(S_0,t_0;\s^i) + \int_{t_0}^T e^{-r(t-t_0)}((\s^r)^2-(\s^h)^2)\frac{S_t^2}{2}\frac{\pa^2 C}{\pa S^2}(S_t,t;\s^h) dt    
    \]
\end{theorem}

\begin{proof}
    Similar with previous proof, the instantaneous profit of the hedged position is
    \begin{eqnarray}
        d\pi_t &=& dC^i - \Delta^h dS - (C^i -\Delta^h S)rdt  \nonumber \\
        &=& \prt{dC^i - dC^h - r(C^i - C^h)dt} + \prt{dC^h - \Delta^h dS  - r(C^h-\Delta^h dS)dt}  
        \label{hedging_dynamic_const_vol_raw}
    \end{eqnarray}

    From proofs in Theorem(\ref{theorem_hedging_real_vol}) , we can see that the present values of first term in equation(\ref{hedging_dynamic_const_vol_raw}) equals
    \[
        C^h_{t_0} - C^i_{t_0}    
    \]

    From proofs in Theorem(\ref{theorem_hedging_imp_vol}) , the present value of second item in equation(\ref{hedging_dynamic_const_vol_raw}) arrives
    \[
        \frac{1}{2}\prt{(\s^i)^2 - (\s^r)^2 }  \int_{t_0}^T e^{-r(t-t_0)}  S^2_t \Gamma^i dt  
    \]
    
    Therefore the present value of total P$\&$L is 
    \[
        C(S_0,t_0;\s^h)-C(S_0,t_0;\s^i) + \int_{t_0}^T e^{-r(t-t_0)}((\s^r)^2-(\s^h)^2)\frac{S_t^2}{2}\frac{\pa^2 C}{\pa S^2}(S_t,t;\s^h) dt   
    \]

\end{proof}


\section{Local volatility models}

In section(\ref{sect_BS}) we see that the Black-Scholes model with constant volatility
cannot reproduce all the option prices observed in the market for a given underlying
because their implied volatility varies with strike and maturity.

To take
into account the market implied volatility smile while staying within a Markovian
and complete model (one risk factor), a natural solution is to model the
volatility as a deterministic function of time and the value of the underlying
\begin{eqnarray}
  \frac{dS_t}{S_t} = rdt + \s(t, S_t)dW_t  
  \label{local_vol_sde}
\end{eqnarray}
where $r$ is the interest rate, assumed to be constant, and $W_t$ is the Brownian motion
under the risk-neutral measure $Q$. The equation(\ref{local_vol_sde}) defines a \emph{local volatility
model}.


By the same procedure of building self-financing portfolio, shown in section(\ref{sect_BS}), we see that the price of an option with payoff $h(S_T)$ at date date $T$ is given by 
\[
    C(t,S_t) = \E^Q [ e^{-r(T-t)}h(S_T) \vert S_t  ]    
\]
and is characterized by the partial differential equation
\[
    rC(t,S) = \frac{\pa C}{\pa t} +rS\frac{\pa C}{\pa S} + \frac{1}{2} \s(t,S)^2 S^2 \frac{\pa^2 C}{\pa S^2}, \quad C(T,S)=h(S)    
\]

The pricing equation has the same form
as in the Black-Scholes model, but one can no longer deduce an explicit pricing
formula, because the volatility is now a function of the underlying.


Practicioners have adopted a \emph{model calibration} approach which allows
to use all observed prices of quoted options as an input for their pricing and
hedging activities. The parameters
obtained by calibration are of course different from those which would be
obtained by historical estimation. But this does not imply any problem related
to the presence of arbitrage opportunities.


The model calibration approach is adopted in view of the fact that financial
markets do not obey to any fundamental law except the simplest no-dominance
or the slightly stronger no-arbitrage. There is no universally
accurate model in finance, and any proposed model is wrong. Therefore,
practitioners primarily base their strategies on comparison between assets, this
is exactly what calibration does.

\subsection{Dupire's formula}

\begin{theorem}
If $S_t$ follows
\[
    \frac{dS_t}{S_t} = rdt + \s(t, S_t)dW_t, S_{t_0} = S_0      
\]
\end{theorem}
and suppose that
\begin{enumerate}
    \item $S_t$ is square integrable, i.e. $\E [ \int_{t_0}^T S_t^2 dt ] < \infty$, $\forall T$.
    \item For $\forall t > t_0$, $S_t$ has a density $p(t, x)$ which is continue on $(t_0, \infty) \times (0, \infty)$.
    \item Diffusion coeffient $\s(t, x)$ is continue on $(t_0, \infty) \times (0, \infty)$.
\end{enumerate}

Then the value of the call option $C(T, K)=e^{-r(T-t)}\E [(S_T - K)^{+}]$ satisfies Dupire's equation
\[
    \frac{\pa C}{\pa T} = \frac{\s(T,K)^2 K^2}{2} \frac{\pa^2 C}{\pa K^2} - rK\frac{\pa C}{\pa K} ,  \quad (T,K) \in [t_0, \infty) \times [0, \infty) 
\] 
with the initial condition $C(t_0, K) = (S_0 - K)^+$

\begin{proof}
 Please see section (10.3) in Touzi \cite{Touzi}. 
\end{proof}


\subsubsection{Calibration of local volatility models}
Modelling local volatility by Dupire's formula is criticized for: 
\begin{itemize}
    \item Market prices are not known for all strikes and all maturities. They must
    be interpolated and the final result is very sensitive to the interpolation
    method used.
    \item The only risk factor is underlying price and therefore is impossible to incorporate 
    volatility risk.
    \item Because of the need to calculate the second derivative of the option price
    function $C(T, K)$, small data errors lead to very large errors in the solution
    (ill-posed problem).
\end{itemize}

Due to these two problems, in practice, Dupire's formula is not used directly
on the market prices. To avoid solving the ill-posed problem, practitioners
typically use one of two approaches to calibrate the model:
\begin{enumerate}
    \item Start by a preliminary calibration of a parametric functional form to the
    implied volatility surface (for example, a function quadratic in strike and
    exponential in time may be used). With this smooth parametric function,
    recalculate option prices for all strikes, which are then used to calculate
    the local volatility by Dupire's formula.
    \item Reformulate Dupire's equation as an optimization problem by introducing
    a penalty term to limit the oscillations of the volatility surface.
\end{enumerate}


\section{Stochastic volatility models}

\begin{eqnarray*}
    &&\frac{d S_t}{S_t} = \mu_t dt + \s_t dW_t   \\
    &&d \s_t = a_t dt + b_t d W^{'}_t, \quad d<W, W^{'}>=\rho dt
\end{eqnarray*}
where $a_t=a(t, \s_t, S_t), b_t=b(t, \s_t, S_t)$, $\rho \in (-1, 1)$. Because there are two source of risks, two underlyings are desired for hedging the risks. 

Assume there exists a liquid underlying with the price
\[
    C_t^0=C^0(t, \s_t, S_t)    
\]
where the deterministic function $C^0(t, \s_t, S_t)$ is known and $\frac{\pa C^0(t, \s, S)}{\pa \s} > 0$ for all $t,\s, S$. Besides some assumptions of assuring the
existence and regularity of the solution has to be made but not listed in details here.

Use $V_t$ to denote the value of a self-financing portfolio with $\de_t$ stocks and $\om_t$ quantity of $C^0$. By self-financing condition and I't$\hat{o}$ lemma, we have
\begin{eqnarray}
    dV_t &&= \prt{V_t - \de_tS_t - \om_tC^0_t} rdt + \de_tS_t + \om_tC^0_t \nonumber \\
    && =  \prt{V_t - \de_tS_t - \om_tC^0_t} rdt + \prt{\de_t + \om_t \frac{\pa C^0}{\pa S}}dS_t + \om_t \frac{\pa C^0}{\pa \s} d\s_t + \om_t {\cal{L}}_t C^0 dt
    \label{sto_vol_self_financing}
\end{eqnarray}
where 
\[
    {\cal{L}}_t = \frac{\pa}{\pa t} + \frac{1}{2} S_t^2 \s_t^2 \frac{\pa^2}{\pa S^2} + \frac{1}{2} b_t^2 \s_t^2 \frac{\pa^2}{\pa \s^2} + S_t\s_t b_t\rho \frac{\pa^2}{\pa S \pa \s}
\]

Suppose this self-financing portfolio replicates a deterministic payoff function $C: V_t =C(t, \s_t, S_t)$. By I't$\hat{o}$ lemma, it is obtained that
\[
    dV_t = {\cal{L}}_tC dt = {\cal{L}}_t C dt + \frac{\pa C}{ \pa S}dS_t + \frac{\pa C}{ \pa \s}d\s_t 
\]

To equalize the above dynamics and equation(\ref{sto_vol_self_financing}), we arrive 
\begin{eqnarray*}
    && \om_t = \frac{\pa C / \pa \s}{\pa C^0 / \pa \s}   \\
    && \s_t = \frac{\pa C}{\pa S} - \om_t \frac{\pa C^0}{\pa S}\\
    && {\cal{L}}_tC - rC + rS_t \frac{\pa C}{\pa S} = \frac{\pa C}{\pa \s}  \frac{{\cal{L}}_t C^0 - rC^0 + rS_t \frac{\pa C^0}{\pa S}}{\pa C^0 / \pa \s}
\end{eqnarray*}

Set $\lambda = - \frac{{\cal{L}}_t C^0 - rC^0 + rS_t \frac{\pa C^0}{\pa S}}{\pa C^0 / \pa \s}$, note that $\lambda$ does not depend on values of option that we are going to hedge, but 
only on values of the stock that has been chosen initially. By replication we arrive at the equation that the price of the option with payoff $h(S_T)$ satisfies
\[
    {\cal{L}}_tC - rC + rS_t \frac{\pa C}{\pa S} + \lambda(t, \s, S) \frac{\pa C}{\pa \s}= 0, \quad C(T, \s, S) = h(S)    
\] 
where 
\[
    {\cal{L}}_t = \frac{\pa}{\pa t} + \frac{1}{2} S_t^2 \s_t^2 \frac{\pa^2}{\pa S^2} + \frac{1}{2} b_t^2 \s_t^2 \frac{\pa^2}{\pa \s^2} + S_t\s_t b_t\rho \frac{\pa^2}{\pa S \pa \s}   
\]

This is the generalization of the Black-Scholes equation with stochastic volatility. The hedging strategy with two types of underlyings whose hedging ratio satisfying $\om_t = \frac{\pa C / \pa \s}{\pa C^0 / \pa \s}$ and 
$\s_t = \frac{\pa C}{\pa S} - \om_t \frac{\pa C^0}{\pa S}$ is called delta-vega hedging(risk of vega is risk of volatility).


\subsection{Pricing formula of caplet and calibration to caps market}

Consider the Libor rate for time period $[T_s, T_e]$, which is fixed at time $t_f \le T_s$. The market quotes give Black volatilities of caplets for various strikes. Suppose a caplet has strike $K$ and the year fraction of $[T_s, T_e]$ is $\tau$. The market price for this caplet is
\[
V_0^{mkt} = P(0,T_e)\tau \mbox{Bl}(K, F, \sigma_B\sqrt{t_f}, 1)
\]
where $F=F(0;T_s, T_e)$ is the forward rate for $[T_s, T_e]$ at time $0$, $\sigma_B$ is the market quoted Black vol, and $\mbox{Bl}(K,F,v,w)$ is given by
\[
\mbox{Bl}(K,F,v,w)=Fw\Phi(wd_1) - Kw\Phi(wd_2), \; d_1 = \frac{\ln(F/K)+v^2/2}{v}, \; d_2 = \frac{\ln(F/K)-v^2/2}{v}
\]

The theoretical price of the above caplet based on one-factor LGM model is
\[
V_0^{model} = P(0,T_e)\tau \mbox{Bl}\left(K+\frac{1}{\tau}, F+\frac{1}{\tau}, \left[H(T_e)-H(T_s)\right] \sqrt{\zeta_{t_f}}, 1\right)
\]
where $H(t) = \int_0^t e^{-\kappa s}ds$ and $\zeta_t = \int_0^t e^{2\kappa s}\sigma_s^2 ds$ ($\sigma_{\cdot}$ is the volatility parameter in the corresponding one-factor Hull-White model).

Therefore, calibration to caplets in order to obtain $\zeta_{t_f}$ requires solving the following equation
\[
\boxed{
\mbox{Bl}(K, F, \sigma_B\sqrt{t_f}, 1) = \mbox{Bl}\left(K+\frac{1}{\tau}, F+\frac{1}{\tau}, \left[H(T_e)-H(T_s)\right] \sqrt{\zeta_{t_f}}, 1\right)
}
\]
We note $\mbox{Bl}(K, F, v, 1)$ is a monotone increasing function of $v$ with a range of $((F-K)^+, F)$. So the above calibration equation always has a solution.

\subsection{Interpolation of LGM model parameter $\zeta$}

We assume there is a coupon period $[t_s, t_e]$ and we are given the values of $\zeta$ at $t_s$ and $t_e$:
$\zeta_s$ and $\zeta_e$, respectively. If the volatility of one-factor Hull-White model (equivalent to one-factor
LGM model) is a constant $\sigma$ over $[t_s, t_e]$, we have for $t\in [t_s, t_e]$
\[
\zeta_t =
\begin{cases}
\zeta_s + \sigma^2(t-t_s) & \mbox{if $\kappa=0$} \\
\zeta_s + \sigma^2 \frac{e^{2\kappa t} - e^{2\kappa t_s}}{2\kappa} & \mbox{if $\kappa \ne 0$}
\end{cases}
\]

By setting $t$ to $t_e$, we can solve for $\sigma$:
\[
\sigma =
\begin{cases}
\frac{\zeta_s(t_e-t)+\zeta_e(t-t_s)}{t_e-t_s} & \mbox{if $\kappa=0$} \\
\frac{e^{2\kappa(t_e-t_s)} - e^{2\kappa(t-t_s)}}{e^{2\kappa(t_e-t_s)}-1}\zeta_s + \frac{e^{2\kappa(t-t_s)}-1}{e^{2\kappa(t_e-t_s)}-1}\zeta_e & \mbox{if $\kappa \ne 0$}
\end{cases}
\]

\section{Elements of two-factor LGM model}\label{sect_LGM2F}

\subsubsection{Approximate formula}

Same as the one-factor case, the pricing formula of swaption can be written as
\[
V^{opt}_{rec}(0) = L_0 E^{Q_L}[(K^{adj}- S(t_{ex}))^+],
\]
where
\[
L(t) = \sum_{i=1}^n\tau_i P(t,t_i)
\]
($t\le t_0$, $i=1,\cdots, n$) is the annuity numeraire,
$S(t)=\frac{P(t,t_0)-P(t,t_n)}{\sum_{i=1}^n\tau_iP(t,t_i)}$ is the
forward swap rate, and $Q_L$ is the martingale measure associated with
annuity numeraire. By the equivalent vol technique, we have (see Hagan
\cite{Haganb}, formula(3.66a))
\begin{eqnarray}\label{formula_HW2F_calib_normal_vol}
\boxed{
\sigma_N^2t_{ex} = {\bf H}_{tot} \cdot \zeta {\bf H}_{tot}
}
\end{eqnarray}
where
\[
{\bf H}_{tot} = \frac{S(0)\sum_{i=1}^n \tau_i D_i \Delta {\bf H}_i +
D_n\Delta {\bf H}_n}{\sum_{i=1}^n\tau_iD_i}
\]
and $\Delta {\bf H}_i = {\bf H}_i - {\bf H}_0$. This allows us to
obtain the swaption price by Black's formula with normal vol $\sigma_N
= \sqrt{\frac{{\bf H}_{tot}\cdot \zeta_{t_{ex}}{\bf
H}_{tot}}{t_{ex}}}$:
\begin{eqnarray}\label{formula_HW2F_swaption_approximate}
\boxed{
V_{rec}^{opt}(0) = L_0
\left[(K^{adj}-S(0))\Phi(-d_1)+\frac{\sigma_N\sqrt{t_{ex}}}{\sqrt{2\pi}}e^{-d_1^2/2}\right]
}
\end{eqnarray}
where $\Phi(\cdot)$ is the c.d.f. of standard normal distribution and
$d_1 = \frac{S(0)-K^{adj}}{\sigma_N \sqrt{t_{ex}}}$.

\subsection{Calibration to ATM swaption market}

For a given sequence of expiries $t_{ex}^0 < t_{ex}^1 < \cdots <
t_{ex}^N$, we shall use the approximate formula
(\ref{formula_HW2F_calib_normal_vol}) to obtain $\zeta(t^i_{ex})$,
$i=1,\cdots, N$. In the case of one-factor model, formula
(\ref{formula_HW2F_calib_normal_vol}) alone is able to produce
$\zeta_{t_{ex}}$; in the case of two-factor model, formula
(\ref{formula_HW2F_calib_normal_vol}) is one equation for two unknowns
(assuming $\alpha$ or $\sigma$ is piecewise constant). So we typically
need a sequence of swaptions to deduce $(\zeta_{t^i_{ex}})_{i=1}^N$ by
bootstrapping.

More precisely, suppose the tenors are $T_1$, $T_2$, $\cdots$, $T_M$.
Formula (\ref{formula_HW2F_calib_normal_vol}) gives us
\[
(\sigma_N^{ij})^2 t^i_{ex} - {\bf H}^{ij}_{tot} \cdot
\zeta_{t^{i-1}_{ex}} {\bf H}^{ij}_{tot} = {\bf H}^{ij}_{tot} \cdot
(\zeta_{t^i_{ex}} - \zeta_{t^{i-1}_{ex}}) {\bf H}^{ij}_{tot},
\]
where ${\bf H}^{ij}_{tot}$ is the quantity similar to the one in
formula (\ref{formula_HW2F_calib_normal_vol}) and corresponds to the
$t_{ex}^i\times T^j$ swaption, and $\sigma_N^{ij}$ is the normal vol
for the $t_{ex}^i \times T^j$ swaption. Then $\zeta_{t^i_{ex}}$ is
defined via $\alpha$ or $\sigma$ (recall $\zeta_t =
\int_0^t\alpha_s^2ds=\int_0^te^{2\kappa s}\sigma_s^2ds$) such that
\[
\sum_{j=1}^M \left\{{\bf H}^{ij}_{tot} \cdot (\zeta_{t^i_{ex}} -
\zeta_{t^{i-1}_{ex}}) {\bf H}^{ij}_{tot} - \left[(\sigma_N^{ij})^2
t^i_{ex} - {\bf H}^{ij}_{tot} \cdot \zeta_{t^{i-1}_{ex}} {\bf
H}^{ij}_{tot}\right]\right\}^2\omega_{ij}^2
\]
is minimized (assuming $\zeta_{t_{ex}^{i-1}}$ is already determined).
Here $(\omega_{ij})_{0\le i \le N, 1\le j \le M}$ is a weight matrix.
To represent $\zeta_t$ in terms of $\alpha$ or $\sigma$, we note for
$s<t$,
\[
\zeta_t - \zeta_s =
\begin{cases}
\left(\begin{matrix} \alpha_1^2(t-s) & \rho \alpha_1\alpha_2(t-s) \\
\rho \alpha_1\alpha_2(t-s) & \alpha_2^2(t-s) \end{matrix} \right)&
\mbox{if $\alpha$ is constant in $[s,t]$} \\
\left(\begin{matrix}
\sigma_1^2\frac{e^{-2\kappa_1s}-e^{-2\kappa_1t}}{2\kappa_1} & \rho
\sigma_1\sigma_2\frac{e^{-(\kappa_1+\kappa_2)s} -
e^{-(\kappa_1+\kappa_2)t}}{\kappa_1+\kappa_2} \\ \rho \sigma_1\sigma_2
\frac{e^{-(\kappa_1+\kappa_2)s} -
e^{-(\kappa_1+\kappa_2)t}}{\kappa_1+\kappa_2} &
\sigma_2^2\frac{e^{-2\kappa_2s}-e^{-2\kappa_2t}}{2\kappa_2}
\end{matrix} \right) & \mbox{if $\sigma$ is constant in $[s,t]$}
\end{cases}
\]

Then (recall ${\bf H}_{tot}=(H_{tot}^1, H_{tot}^2)$)
\begin{eqnarray*}
& & {\bf H}_{tot}\cdot (\zeta_t - \zeta_s) {\bf H}_{tot} \\
&=& (H^1_{tot})^2(\zeta_t-\zeta_s)_{11} + 2 H^1_{tot}
H^2_{tot}(\zeta_t-\zeta_s)_{12} + (H^2_{tot})^2(\zeta_t-\zeta_s)_{22}
\\
&=&
\begin{cases}
\left[(H_{tot}^1)^2\alpha_1^2 + 2 H_{tot}^1
H_{tot}^2\rho\alpha_1\alpha_2 + (H_{tot}^2)^2\alpha_2^2 \right](t-s) &
\mbox{if $\alpha$ is constant in $[s,t]$} \\
(H_{tot}^1)^2\frac{e^{-2\kappa_1s}-e^{-2\kappa_1t}}{2\kappa_1}\sigma_1^2
+ 2\rho \Delta H_{tot}^1H_{tot}^2 \frac{e^{-(\kappa_1+\kappa_2)s} -
e^{-(\kappa_1+\kappa_2)t}}{\kappa_1+\kappa_2} \sigma_1\sigma_2 & \\
+ (H_{tot}^2)^2\frac{e^{-2\kappa_2s}-e^{-2\kappa_2t}}{2\kappa_2}\sigma_2^2
& \mbox{if $\sigma$ is constant in $[s,t]$}
\end{cases} \\
&=&
\begin{cases}
\alpha\cdot {\bf A} \alpha & \mbox{if $\alpha$ is constant in $[s,t]$} \\
\sigma\cdot {\bf A} \sigma & \mbox{if $\sigma$ is constant in $[s,t]$}
\end{cases} \\
\end{eqnarray*}
where $\alpha = (\alpha^1, \alpha^2)$, $\sigma = (\sigma_1, \sigma_2)$, and
\[
{\bf A} =
\begin{cases}
\left( \begin{matrix}(H_{tot}^1)^2 & \rho H_{tot}^1H_{tot}^2 \\ \rho
H_{tot}^1H_{tot}^2 & (H_{tot}^2)^2\end{matrix} \right)(t-s) & \mbox{if
$\alpha$ is constant in $[s,t]$} \\
\left(\begin{matrix}
(H_{tot}^1)^2\frac{e^{-2\kappa_1s}-e^{-2\kappa_1t}}{2\kappa_1} & \rho
H_{tot}^1H_{tot}^2\frac{e^{-(\kappa_1+\kappa_2)s} -
e^{-(\kappa_1+\kappa_2)t}}{\kappa_1+\kappa_2} \\ \rho
H_{tot}^1H_{tot}^2 \frac{e^{-(\kappa_1+\kappa_2)s} -
e^{-(\kappa_1+\kappa_2)t}}{\kappa_1+\kappa_2} &
(H_{tot}^2)^2\frac{e^{-2\kappa_2s}-e^{-2\kappa_2t}}{2\kappa_2}
\end{matrix} \right) & \mbox{if $\sigma$ is constant in $[s,t]$}
\end{cases}
\]

Therefore, the calibration problem is reformulated to the following
optimization problem:
\begin{eqnarray}\label{formula_HW2F_calib_optimization}
\boxed{
\begin{cases}
\arg_{\alpha_i}\min \sum_{j=1}^M \left\{\alpha_i\cdot {\bf
A}^{ij}\alpha_i -\left[(\sigma_N^{ij})^2 t_{ex}^i- {\bf H}^{ij}_{tot}
\cdot \zeta_{t^{i-1}_{ex}} {\bf
H}^{ij}_{tot}\right]\right\}^2\omega^2_{ij}, \; (i=1,\cdots,N) &
\mbox{if $\alpha$ is piecewise constant} \\
\arg_{\sigma_i}\min \sum_{j=1}^M \left\{\sigma_i\cdot {\bf
A}^{ij}\sigma_i -\left[(\sigma_N^{ij})^2 t_{ex}^i- {\bf H}^{ij}_{tot}
\cdot \zeta_{t^{i-1}_{ex}} {\bf
H}^{ij}_{tot}\right]\right\}^2\omega^2_{ij}, \; (i=1,\cdots,N) &
\mbox{if $\sigma$ is piecewise constant}
\end{cases}
}
\end{eqnarray}
where for the $i$-th optimization problem to be solved, the first
$(i-1)$ optimization problems must be already solved.

\subsection{Calibration to CMS spread option}

The calibration of two-factor LGM model to CMS spread option is based
on a sequence of swaptions  and a sequence of CMS spread options.

More precisely, for a given sequence of expiries $t_{ex}^0 < t_{ex}^1
< \cdots < t_{ex}^N$ and corresponding sequence of swap maturities
$T_1, T_2, \cdots, T_N$, formula (\ref{formula_HW2F_calib_normal_vol})
gives
\[
(\sigma_N^i)^2 t_{ex}^i - {\bf H}_{tot}^i \cdot \zeta_{t_{ex}^{i-1}}
{\bf H}_{tot}^i = {\bf H}_{tot}^i \cdot (\zeta_{t_{ex}^{i}} -
\zeta_{t_{ex}^{i-1}}) {\bf H}_{tot}^i, \; i=1,\cdots, N,
\]
where $\sigma_N^i$ is the normal vol for the $t^i_{ex} \times T_i$
swaption and ${\bf H}_{tot}^i$ is the quantity similar to the one in
formula (\ref{formula_HW2F_calib_normal_vol}) corresponding to the
$t_{ex}^i \times T^i$ swaption.

Assuming $\zeta_{t_{ex}^1}$, $\cdots$, $\zeta_{t_{ex}^{i-1}}$ have
been given, to find out $\zeta_{t_{ex}^i}$, we repeat what's done in
ATM calibration and write ${\bf H}_{tot}^i \cdot (\zeta_{t_{ex}^{i-1}}
- \zeta_{t_{ex}^i}) {\bf H}_{tot}^i$ as
\[
{\bf H}_{tot}^i \cdot (\zeta_{t_{ex}^{i}} - \zeta_{t_{ex}^{i-1}}) {\bf
H}_{tot}^i = \begin{cases}
\alpha \cdot {\bf A}_i \alpha & \mbox{if $\alpha$ is constant in
$[t_{ex}^{i-1}, t_{ex}^i]$} \\
\sigma \cdot {\bf A}_i \sigma & \mbox{if $\sigma$ is constant in
$[t_{ex}^{i-1}, t_{ex}^i]$} \\
\end{cases}
\]
We further parameterize ${\bf A}_i$ by finding $r_1$, $r_2$, $\phi_0$ such that
\[
\frac{{\bf A}_i}{(\sigma_N^i)^2t_{ex}^i - {\bf H}_{tot}^i \cdot
\zeta_{t_{ex}^{i-1}}{\bf H}_{tot}^i} =
\left(\begin{matrix}\cos\phi_0 & -\sin\phi_0 \\ \sin\phi_0 &
\cos\phi_0 \end{matrix} \right)
\left(\begin{matrix}\frac{1}{r_1^2} & 0 \\ 0 & \frac{1}{r_2^2}
\end{matrix} \right)
\left(\begin{matrix}\cos\phi_0 & \sin\phi_0 \\ -\sin\phi_0 &
\cos\phi_0 \end{matrix} \right)
\]
Then the first equation for $\zeta_{t_{ex}^i}$ is
\[
1 = \begin{cases}
\alpha \cdot \left(\begin{matrix}\cos\phi_0 & -\sin\phi_0 \\
\sin\phi_0 & \cos\phi_0 \end{matrix} \right)
\left(\begin{matrix}\frac{1}{r_1^2} & 0 \\ 0 & \frac{1}{r_2^2}
\end{matrix} \right)
\left(\begin{matrix}\cos\phi_0 & \sin\phi_0 \\ -\sin\phi_0 &
\cos\phi_0 \end{matrix} \right) \alpha & \mbox{if $\alpha$ is constant
in $[t_{ex}^{i-1}, t_{ex}^i]$} \\
\sigma \cdot \left(\begin{matrix}\cos\phi_0 & -\sin\phi_0 \\
\sin\phi_0 & \cos\phi_0 \end{matrix} \right)
\left(\begin{matrix}\frac{1}{r_1^2} & 0 \\ 0 & \frac{1}{r_2^2}
\end{matrix} \right)
\left(\begin{matrix}\cos\phi_0 & \sin\phi_0 \\ -\sin\phi_0 &
\cos\phi_0 \end{matrix} \right) \sigma & \mbox{if $\sigma$ is constant
in $[t_{ex}^{i-1}, t_{ex}^i]$} \\
\end{cases}
\]

For the second equation, consider an ATM CMS spread option with payoff
\[
[S_1(t_{ex}^i)-S_2(t_{ex}^i)-K]^+
\]
at time $t^i_{ex}$, where $S_1$ and $S_2$ are two swap rates, and $K =
E^{Q_{t_{ex}^i}}[S_1(t_{ex}^i)] - E^{Q_{t_{ex}^i}}[S_2(t_{ex}^i)]$.
The normal vol of $S_1(t_{ex}^i)-S_2(t_{ex}^i)$ can be quoted directly
from market, while Hagan's formula gives the normal spread vol as
\[
(\sigma_N^{sprd})^2 t_{ex}^i = ({\bf H}_{tot}^1 - {\bf H}_{tot}^2)
\cdot \zeta_{t_{ex}^i} ({\bf H}_{tot}^1 - {\bf H}_{tot}^2),
\]
where ${\bf H}_{tot}^1$ corresponds to $S_1$ and ${\bf H}_{tot}^2$
corresponds to $S_2$. Matching $\sigma_N^{sprd}$ with market quote
will give us the second equation for $\zeta_{t^i_{ex}}$.

\medskip

These two equations combined allow us to solve for $\zeta_{t^i_{ex}}$.
More precisely, we use $x = (x_1, x_2)^T$ to stand for either $\alpha$
or $\sigma$. On the interval $[t_{ex}^{i-1}, t_{ex}^i]$ we have two
equations for two unknowns:
\begin{eqnarray}\label{eqn_alph1_alpha2}
\begin{cases}
1 = (x_1,x_2) \cdot \left(\begin{matrix}\cos\phi_0 & -\sin\phi_0 \\
\sin\phi_0 & \cos\phi_0 \end{matrix} \right)
\left(\begin{matrix}\frac{1}{r_1^2} & 0 \\ 0 & \frac{1}{r_2^2}
\end{matrix} \right)
\left(\begin{matrix}\cos\phi_0 & \sin\phi_0 \\ -\sin\phi_0 &
\cos\phi_0 \end{matrix} \right) \left(\begin{matrix} x_1 \\ x_2
\end{matrix} \right) \\
(\sigma_N^{mkt})^2 t^i_{ex} = ({\bf H}_{tot}^1 - {\bf H}_{tot}^2)
\cdot \zeta_{t_{ex}^i} ({\bf H}_{tot}^1 - {\bf H}_{tot}^2)
\end{cases}
\end{eqnarray}
where $\sigma_N^{mkt}$ is market quote of the normal spread vol and
\begin{eqnarray}\label{formula_alpha2zeta}
\zeta_{t^i_{ex}} = \zeta_{t^{i-1}_{ex}} + \left( \begin{matrix} x_1 &
0 \\ 0 & x_2 \end{matrix}\right)  {\bf B} \left( \begin{matrix} x_1 &
0 \\ 0 & x_2 \end{matrix}\right).
\end{eqnarray}
where
\[
{\bf B} =
\begin{cases}
\left( \begin{matrix}1 & \rho  \\ \rho & 1\end{matrix}
\right)(t_{ex}^i-t_{ex}^{i-1}) & \mbox{if $\alpha$ is constant in
$[t_{ex}^{i-1},t_{ex}^i]$} \\
\left(\begin{matrix} \frac{e^{-2\kappa_1
t_{ex}^{i-1}}-e^{-2\kappa_1t_{ex}^i}}{2\kappa_1} & \rho
\frac{e^{-(\kappa_1+\kappa_2)t_{ex}^{i-1}} -
e^{-(\kappa_1+\kappa_2)t_{ex}^i}}{\kappa_1+\kappa_2} \\ \rho
\frac{e^{-(\kappa_1+\kappa_2)t_{ex}^{i-1}} -
e^{-(\kappa_1+\kappa_2)t_{ex}^i}}{\kappa_1+\kappa_2} &
\frac{e^{-2\kappa_2t_{ex}^{i-1}}-e^{-2\kappa_2t_{ex}^i}}{2\kappa_2}
\end{matrix} \right) & \mbox{if $\sigma$ is constant in
$[t_{ex}^{i-1},t_{ex}^i]$}
\end{cases}
\]

We set the re-parametrization
\[
\begin{cases}
x_1 = r\cos\phi \\
x_2 = r\sin\phi
\end{cases}
\]
such that the first equation of system (\ref{eqn_alph1_alpha2}) becomes
$1 = \frac{r^2 \cos^2(\phi-\phi_0)}{r_1^2} +
\frac{r^2\sin^2(\phi-\phi_0)}{r_2^2}$, or equivalently,
\begin{eqnarray}\label{formula_phi2r}
r = \frac{r_1r_2}{\sqrt{r_2^2\cos^2(\phi-\phi_0)+r_1^2\sin^2(\phi-\phi_0)}}.
\end{eqnarray}
This allows us to use the following trial-and-error procedure to solve
system (\ref{eqn_alph1_alpha2}):
\newline Step 1. try a testing value of $\phi$;
\newline Step 2. use formula (\ref{formula_phi2r}) to obtain the value of $r$;
\newline Step 3. obtain the values of $x_1$ and $x_2$ by those of $r$
and $\phi$;
\newline Step 4. obtain $\zeta_{t^i_{ex}}$ by formula
(\ref{formula_alpha2zeta});
\newline Step 5. use the RHS of the second equation of system
(\ref{eqn_alph1_alpha2}) to obtain theoretical normal spread vol
$\sigma_N^{sprd}$;
\newline Step 6. check the error $|\sigma_N^{sprd} - \sigma_N^{mkt}|$:
if sufficently small, stop; otherwise, return to Step 1 with a
different value for $\phi$.

The above procedure has the advantage of reducing problem's
dimensionality. The cost is a complicated nonlinear equation of the
unknown $\phi$, such that it becomes hard to analyze the effectiveness
of global Newton's method:
\[
\frac{(\sigma_N^{mkt})^2 t_{ex}^i - ({\bf H}_{tot}^{1} - {\bf
H}_{tot}^{2}) \cdot \zeta_{t_{ex}^{i-1}}({\bf H}_{tot}^{1} - {\bf
H}_{tot}^{2})}{r_1^2 r_2^2} = \frac{({\bf H}_{tot}^{1} - {\bf
H}_{tot}^{2}) \cdot \left(\begin{matrix} \cos\phi & 0 \\ 0 & \sin\phi
\end{matrix} \right) {\bf B} \left(\begin{matrix} \cos\phi & 0 \\ 0 &
\sin\phi \end{matrix} \right)({\bf H}_{tot}^{1} - {\bf
H}_{tot}^{2})}{r_2^2\cos^2(\phi-\phi_0)+r_1^2\sin^2(\phi-\phi_0)}
\]


This equation can actually be simplified to give closed-form solution
as follows. We define ${\bf h} = {\bf H}_{tot}^{1} - {\bf H}_{tot}^{2}
= (h_1, h_2)^T$ and ${\bf Q} = \left(\begin{matrix} h_1 & 0 \\ 0 & h_2
\end{matrix}\right) {\bf B} \left(\begin{matrix} h_1 & 0 \\ 0 & h_2
\end{matrix}\right)$. Then the equation becomes
\begin{eqnarray}\label{eqn_phi}
\frac{(\sigma_N^{mkt})^2 t_{ex}^i - {\bf h} \cdot
\zeta_{t_{ex}^{i-1}}{\bf h} }{r_1^2 r_2^2} =
\frac{\left(\begin{matrix} \cos\phi & \sin\phi \end{matrix}\right){\bf
Q} \left(\begin{matrix} \cos\phi \\ \sin\phi
\end{matrix}\right)}{r_2^2\cos^2(\phi-\phi_0)+r_1^2\sin^2(\phi-\phi_0)}
\end{eqnarray}

To solve for $\phi$, we set $C=(\sigma_N^{mkt})^2 t_{ex}^i - {\bf h}
\cdot \zeta_{t_{ex}^{i-1}}{\bf h}$, $\theta = \phi - \phi_0$, and
\begin{eqnarray*}
\hat {\bf Q} &=& \left(\begin{matrix}\cos\phi_0 & \sin\phi_0 \\
-\sin\phi_0 & \cos\phi_0 \end{matrix}\right) {\bf Q}
\left(\begin{matrix}\cos\phi_0 & -\sin\phi_0 \\ \sin\phi_0 &
\cos\phi_0 \end{matrix}\right) \\
&=&
\left(
\begin{matrix}
Q_{11}\cos^2\phi_0 + Q_{22} \sin^2\phi_0 + 2Q_{12}\sin\phi_0\cos\phi_0
& (Q_{22}-Q_{11})\sin\phi_0\cos\phi_0 +
Q_{12}(\cos^2\phi_0-\sin^2\phi_0)\\
(Q_{22}-Q_{11})\sin\phi_0\cos\phi_0 +
Q_{12}(\cos^2\phi_0-\sin^2\phi_0) & Q_{11}\sin^2\phi_0 +
Q_{22}\cos^2\phi_0 - 2Q_{12}\sin\phi_0\cos\phi_0
\end{matrix}
\right)
\end{eqnarray*}
Then the equation (\ref{eqn_phi}) becomes
\begin{eqnarray}\label{eqn_theta}
\sin(2\theta + \gamma) = \frac{C(r_1^2+r_2^2)-r_1^2r_2^2(\hat Q_{11} +
\hat Q_{22})}{\sqrt{[r_1^2r_2^2(\hat Q_{11}-\hat Q_{22}) - C(r_2^2 -
r_1^2)]^2+ 4r_1^4r_2^4\hat Q_{12}^2}}
\end{eqnarray}
where $\gamma$ is determined by
\begin{eqnarray}\label{eqn_gamma}
\begin{cases}
\sin\gamma = \frac{r_1^2r_2^2(\hat Q_{11}-\hat Q_{22}) - C(r_2^2 -
r_1^2)}{\sqrt{[r_1^2r_2^2(\hat Q_{11}-\hat Q_{22}) - C(r_2^2 -
r_1^2)]^2+ 4r_1^4r_2^4\hat Q_{12}^2}} \\
\cos\gamma = \frac{2r_1^2r_2^2\hat Q_{12}}{\sqrt{[r_1^2r_2^2(\hat
Q_{11}-\hat Q_{22}) - C(r_2^2 - r_1^2)]^2+ 4r_1^4r_2^4\hat Q_{12}^2}}
\end{cases}
\end{eqnarray}
A necessary and sufficient for equation (\ref{eqn_theta}) to have a solution is
\[
\left| C(r_1^2+r_2^2)-r_1^2r_2^2(\hat Q_{11} + \hat Q_{22}) \right|
\le \sqrt{[r_1^2r_2^2(\hat Q_{11}-\hat Q_{22}) - C(r_2^2 - r_1^2)]^2+
4r_1^4r_2^4\hat Q_{12}^2},
\]
which is equivalent to an inequality for a quadratic polynomial of
$C$. Tedious calculation shows the inequality can be reduced to $C_-
\le C \le C_+$, where
\[
C_{\pm} = \frac{r_1^2\hat Q_{11} + r_2^2 \hat Q_{22}}{2} \pm
\frac{\sqrt{r_1^4\hat Q_{11}^2 + r_2^4\hat Q_{22}^2 + 4r_1^2r_2^2\hat
Q_{12}^2}}{2}.
\]
Therefore, the necessary and sufficient condition for a successful
calibration is
\[
 - \frac{\sqrt{r_1^4\hat Q_{11}^2 + r_2^4\hat Q_{22}^2 +
4r_1^2r_2^2\hat Q_{12}^2}}{2}\le (\sigma_N^{mkt})^2t_{ex}^i - \left[
{\bf h} \cdot \zeta_{t^{i-1}_{ex}} {\bf h} + \frac{r_1^2\hat Q_{11} +
r_2^2 \hat Q_{22}}{2} \right] \le  \frac{\sqrt{r_1^4\hat Q_{11}^2 +
r_2^4\hat Q_{22}^2 + 4r_1^2r_2^2\hat Q_{12}^2}}{2}.
\]

\medskip

Simultaneously, we have discovered the follow procedure to solve
system (\ref{eqn_alph1_alpha2}) explicitly:
\newline Step 1: obtain the value of $C = (\sigma_N^{mkt})^2t^i_{ex} -
{\bf h} \cdot \zeta_{t^{i-1}_{ex}} {\bf h}$ by inserting the value of
$\sigma_N^{mkt}$, and use it to compute the value of $\gamma$ by
equation (\ref{eqn_gamma});
\newline Step 2: invert equation (\ref{eqn_theta}) to obtain the value
of $\theta$;
\newline Step 3: use the value of $\theta$ to obtain the value of
$\phi=\theta+\phi_0$ and compute the value of $r$ by formula
(\ref{formula_phi2r});
\newline Step 4: obtain the values of $x_1$ and $x_2$: $x_1 =
r\cos\phi$, $x_2 = r\sin\phi$;
\newline Step 5: obtain $\zeta_{t_{ex}^i}$ by formula
(\ref{formula_alpha2zeta}).

\begin{remark}
From Step 2, we shall have two representative solutions of $\theta$,
with the relation
\[
\theta^{(1)}+\theta^{(2)} \mod \pi = \frac{\pi}{2}.
\]
This leads to two representative solution pairs of $(r,\phi)$ with the relation
\[
\begin{cases}
\phi^{(1)}+\phi^{(2)} \mod \pi = \frac{\pi}{2} \\
r^{(1)} = \frac{r_1r_2}{\sqrt{r_2^2\cos^2\theta^{(1)} + r_1^2
\sin^2\theta^{(1)}}} = \frac{r_1r_2}{\sqrt{r_2^2\sin^2\theta^{(2)} +
r_1^2 \cos^2\theta^{(2)}}}\\
r^{(2)} = \frac{r_1r_2}{\sqrt{r_2^2\cos^2\theta^{(2)} + r_1^2
\sin^2\theta^{(2)}}} =
\frac{r_1r_2}{\sqrt{r_2^2\sin^2\theta^{(1)} + r_1^2 \cos^2\theta^{(1)}}}\\
\end{cases}
\]
\end{remark}


\begin{appendix}


\end{appendix}

\begin{thebibliography}{99}

\bibitem{Derman} Emanuel Derman and Michael B.Miller. The volatility smile. {\it
Weiley Financial Series}, 2016.

\bibitem{Wilmott} Riaz Ahmad and Paul Wilmott. Which free lunch would you like today, sir? delta hedging, volatility arbitrage and optimal portfolios. {\it
WilMott}, 2005.

\bibitem{Touzi} Nizar Touzi. Calcul Stochatique en Finance. {\it
Lecture Note} {\bf}, PA Math\'ematiques Appliqu\'ees, Ecole Polytechnique, September, 2018.

\bibitem{Tankov} Peter Tankov, Surface de volatilit\'e, {\it
Lecture Note of Master Mod\'elisation Al\'eatoire \`a Paris 7}, ENSAE ParisTech, February, 2015.

\end{thebibliography}

\end{document} 