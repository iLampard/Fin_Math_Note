% LaTeX template, xeCJK usepackage and Unicode text need XeLaTeX to compile.
% MiKTeX package can be downloaded at miktex.org, WinEdt can be downloaded at http://www.winedt.com/.
% WinEdt 6 and higher provide XeLaTeX and Unicode support.

% LaTeX template version 1.0.4, last revised on 2014-11-17.

\documentclass[10pt]{article}



% *************************** installed packages *************************
% AMS packages
\usepackage{amsfonts}   % TeX fonts from the American Mathematical Society.
\usepackage{amsmath}    % AMS mathematical facilities for LaTeX.
\usepackage{amssymb}    % AMS symbols
\usepackage{amsthm}     % Provide proclamations environment.

% graphics packages
\usepackage{graphics}   % Standard LaTeX graphics.
\usepackage{tikz}       % TikZ and PGF package for graphics
\usetikzlibrary{matrix} % matrix library of TikZ package
\usetikzlibrary{trees}  % trees library of TikZ package

% support for foreign languages, esp. Chinese.
\usepackage{xeCJK}      % Support for CJK (Chinese, Japanese, Korean) documents in XeLaTeX.
\setCJKmainfont{SimSun} % MUST appear when xeCJK is loaded.
%\setCJKmainfont{DFKai-SB}          % 设置正文罗马族的CKJ字体,影响 \rmfamily 和 \textrm 的字体。此处设为“标楷体”。
%\setCJKmainfont{SimSun}            % 设置正文罗马族的CKJ字体,影响 \rmfamily 和 \textrm 的字体。此处设为“宋体”。
%\setCJKmonofont{MingLiU}           % 设置正文等宽族的CJK字体,影响 \ttfamily 和 \texttt 的字体。此处设为“细明体”。
%\renewcommand\abstractname{摘要}   % 重定义摘要名:abstract->摘要。
%\renewcommand\appendixname{附录}   % 重定义附录名:appendix->附录。
%\renewcommand\bibname{参考文献}    % 重定义参考文献名:bibliography->参考文献。
%\renewcommand\contentsname{目录}   % 重定义目录名:contents->目录。
%\renewcommand\refname{参考文献}    % 重定义参考文献名:references->参考文献。

% miscellaneous packages
\usepackage[toc, page]{appendix}   % Extra control of appendices.
\usepackage{caption}    % Customising captions in floating environments.
%\captionsetup[figure]{labelformat=empty} % redefines the caption setup of the figures environment in the beamer %class.
\usepackage{clrscode}   % Typesets pseudocode as in Introduction to Algorithms.
\usepackage{epsfig}
\usepackage{eurosym}    % Metafont and macros for Euro sign.
\usepackage{float}      % Improved interface for floating objects.
\usepackage{fontspec}   % Advanced font selection in XeLaTeX and LuaLaTeX.
\usepackage{indentfirst}% Leave no indent for a paragraph after a sectional heading.
\usepackage{xcolor}     % Driver-independent color extensions for LaTeX and pdfLaTeX.

% must-be-the-last packages
\usepackage[driverfallback=hypertex, pagebackref]{hyperref}   % Extensive support for hypertext in LaTeX; MUST be on the last \usepackage line in the preamble. [pagebackref] for page referencing; [backref] for section referencing.
% ********************** end of installed packages ***********************



% ************************** fullpage.sty ********************************
% This is FULLPAGE.STY by H.Partl, Version 2 as of 15 Dec 1988.
% Document Style Option to fill the paper just like Plain TeX.
\typeout{Style Option FULLPAGE Version 2 as of 15 Dec 1988}

\topmargin 0pt \advance \topmargin by -\headheight \advance
\topmargin by -\headsep

\textheight 8.9in

\oddsidemargin 0pt \evensidemargin \oddsidemargin \marginparwidth
0.5in

\textwidth 6.5in
% For users of A4 paper: The above values are suited for American 8.5x11in
% paper. If your output driver performs a conversion for A4 paper, keep
% those values. If your output driver conforms to the TeX standard (1in/1in),
% then you should add the following commands to center the text on A4 paper:

% \advance\hoffset by -3mm  % A4 is narrower.
% \advance\voffset by  8mm  % A4 is taller.
% ************************ end of fullpage.sty ***************************



% ************** Proclamations (theorem-like structures) *****************
% [section] option provides numbering within a section.
\theoremstyle{plain}
\newtheorem{theorem}{Theorem}[section]
\newtheorem{lemma}{Lemma}[section]
\newtheorem{definition}{Definition}[section]
\newtheorem{prop}{Proposition}[section]
\newtheorem{corollary}{Corollary}[section]
\newtheorem{remark}{Remark}
\newtheorem{example}{Example}
\numberwithin{equation}{section}
\numberwithin{table}{section}
% ************************************************************************



% ************* Solutions use a modified proof environment ***************
%\newenvironment{solution}
%  {\begin{proof}[Solution]}
%  {\end{proof}}
% ************************************************************************



% ************* Frequently used commands as shorthand ********************
\newcommand{\norm}{|\!|}


\newcommand{\s}{\sigma}
\newcommand{\om}{\omega}
\newcommand{\prt}[1]{\left( #1 \right)}  % parenthese
\newcommand{\de}{\delta}
\newcommand{\pa}{\partial}
\newcommand{\E}{\mathbb{E}}

% ************************************************************************



\begin{document}

\title{Note on LGM Model}
\author{X}
\date{Version 1.1, 2018-12-07.}

\maketitle

\begin{abstract}
The note documents methodologies of Linear Gaussian Markov model and its applications in 
fixed income product modeling.
\end{abstract}

\tableofcontents

\newpage

\section{HJM framework}
After Ho and Lee (1986) introduced a discrete setting to model the evolution
of the entire yield curve, Heath, Jarrow and Morton (HJM) (1992) developed a
continuous framework for the modeling of interest rate dynamics(see for
instances in \cite{Mercurio}). Precisely they choose the instantaneous
forward rates, that can be locked in at one time for borrowing at a later
time, as the fundamental quantities to model. HJM model is an arbitrage free
framework for the stochastic evolution of the entire yield curve, where the
forward rates dynamics are fully specified through their instantaneous
volatility structures.

\subsection{Forward rates}

Denote $B(t,T)$ as the price at time $t$ of a \emph{zero coupon} bond
maturing at time $T$ and have face value $1$. We assume this bond bears no
risk of default and for every $t$ and $T$, the bond price $B(t,T)$ is
defined.

One can set up a zero cost portfolio, by taking a short position of size 1
in a $T$ maturity bonds and a long position of size $\frac{B(t,T)}{B(t,T +
\delta)}$ in $T + \delta$ maturity bonds. At time $T$ one is required to pay
$1$ to cover the short position and at $T + \delta$ one receives $\frac{%
B(t,T)}{B(t,T + \delta)}$, which is strictly larger than 1.

It can be seen that the yield for the time $[T, T + \delta]$ is determined,
or "locked in", at earlier time $t$. By this meaning, one defines the
instantaneous \emph{forward rate} at time $t$ for investing at time T to be
\[
    f(t,T)=-\underset{\delta \rightarrow 0}{\lim }\frac{\log B(t,T+\delta )-\log
B(t,T)}{\delta }    
\]


More precisely the forward rate is given by
\begin{eqnarray}
f(t,T)=-\frac{\partial }{\partial T}\log B(t,T)  
\label{fwd_def}
\end{eqnarray}

If $f(t,T)$ is known for all values of $t<T$, one can recover $B(0,T)$ by
equation (\ref{fwd_def}) that
\begin{eqnarray}
    B(t,T) = \exp{\left(-\int_{t}^T f(t,u) du \right)}  \label{zc_def}
\end{eqnarray}

Theoretically it does not matter whether to build a model for forward rates
or for bond prices because of equations (\ref{fwd_def}) and (\ref{zc_def}).
In practice forward rates are not easy to directly determine from market
data as it is sensitive to small changes in bond prices.

At last if letting $T=t$ the interest rate at time $t$ is given by
\begin{eqnarray}
r_{t} = f(t,t)  
\label{r_t_def}
\end{eqnarray}
This rate is the instantaneous rate we can lock in at time $t$ for borrowing
at time $t$. It is called spot rate or short rate in practise.


\subsection{Dynamics of forward rates and bond prices}

In the HJM model, forward rates is modeled under the form
\begin{eqnarray}
f(t,T)=f(0,T) + \int_{0}^{t}\gamma(s,T) ds + \int_{0}^{t}\sigma^f(s,T) d
\hat{W}_{s}  \label{HJM_f_def}
\end{eqnarray}
where $\hat{W}_{s}$ is the Brownian motion under the historical probability
measure $P$.

Indeed HJM framework only admits no arbitrage under certain condition where is addressed in
the following theorem.

\begin{theorem}
    \emph{\textbf{(HJM non-arbitrage condition)}} A HJM model
    admits no arbitrage if there exists an adapted process $\theta_{t}$ which
    satisfies
    \[
        \gamma(t,T) = \sigma^f(t,T)\left( \theta_{t} -  \sigma^B(t,T) \right)
    \]
    where $\gamma(t,T)$, $\sigma^f(t,T)$ are the drift and volatility of the
    forward rate under the historical measure and
    \[
        \sigma^B(t,T)= - \int_{t}^T\sigma^f(t,u)\ du
    \]
    \label{thm_hjm_noarb}
\end{theorem}



By applying theorem {\textbf{\ref{thm_hjm_noarb}} and equation (\ref{HJM_f_def}) the dynamics under
the risk neutral measure is given by
\[
    d f(t,T)= - \sigma^f(t,T)\sigma^B(t,T) dt + \sigma^f(t,T) dW_{t}  
\]

where $W_{t}$ is the Brownian motion under the risk neutral measure $Q$. }

Indeed zero coupon is a tradable asset and under risk neutral measure its
drift must be $r_{t}$. It can be shown by using Ito lemma and equation (\ref%
{zc_def}) that
\[
    \frac{d B(t,T)}{ B(t,T)}= r_{t} dt + \sigma^B(t,T) dW_{t}
\]

The following theorem summarizes the discussion.
\begin{theorem}
    In a term-structure model satisfying the HJM no-arbitrage condition stated in theorem(\ref{thm_hjm_noarb}),
    the forward rates evolve according to 
    \begin{eqnarray}
        d f(t,T)= - \sigma^f(t,T)\sigma^B(t,T) dt + \sigma^f(t,T) dW_{t}
    \label{hjm_fwd_def}
    \end{eqnarray}
    and the zero coupon bond prices evolves as 
    \begin{eqnarray}
        \frac{d B(t,T)}{ B(t,T)}= r_{t} dt + \sigma^B(t,T) dW_{t}
    \label{hjm_bond_def}
    \end{eqnarray}
    where $W_t$ is a Brownian motion under a risk-neutral measure $Q$。 Here $\sigma^B(t,T)=-\int_t^T \sigma^f(t,u) du$
    and $r_t = f(t,t)$ is the interest rate. The discounted bond prices satisfy 
    \[
        d \prt{D(0,t) B(t,T)} = D(0,t)\sigma^B(t,T) B(t,T) d W_t    
    \]
    where $D(0,t) = e^{-\int_0^t r_u du}$ is the discount process. The solution to equation(\ref{hjm_bond_def}) is 
    \[
        B(t, T) = B(0,T) \exp\prt{\int_0^t r_u du + \int_0^t \sigma^B(u,T) dW_u - \frac{1}{2}\int_0^t  (\sigma^B(u,T))^2 du  }
    \] 

\end{theorem}

\subsection{Markovian property}

From equation (\ref{r_t_def}) and equation (\ref{hjm_fwd_def}) it can be
verified that under risk neutral measure $Q$ the spot rate is
\[
    r_{t} = f(t,t)=f(0,t)+ \int_{0}^{t}\sigma^f(u,t)\sigma^B(u,t) du +
    \int_{0}^{t}\sigma^f(u,t) dW_{u}
\]

If one focuses on the path dependent term by defining
\[
    G(t) =\int_{0}^{t}\sigma^f(u,t) dW_{u}
\]
Obviously it is generally not Markovian because
\begin{eqnarray*}
\E^Q( G(T)|G(t)) &=& \E^Q \left(G(t)+ \int_{t}^{T}\sigma^f(u,T) dW_{u}
+(\int_{0}^{t}\sigma^f(u,T) dW_{u} - \int_{0}^{t}\sigma^f(u,t) dW_{u}) |
G(t)\right) \\
                &\not=& E^Q_{t}(G(T))
\end{eqnarray*}
unless $\int_{0}^{t}\sigma^f(u,T) dW_{u} - \int_{0}^{t}\sigma^f(u,t) dW_{u}$
is non random or a deterministic function of $G(t)$.

An important area of investigation concerns the conditions under which $%
r_{t} $ is Markov or less strictly, can be written as
\[
    r_{t} = h(t, x_{t})
\]
for a deterministic function $h$ and a finite-dimensional Markovian vector
of state variables $x_{t}$. Following section presents an extended multi
factor Hull-White model that satisfies Markovian property under certain
conditions and is easy to implement in practice.


\section{LGM model}


\subsection{Model setup}
A Linear Gaussian Markov(LGM) model assumes
that the volatility $\sigma^f(t,T)$ of the forward rate $f(t,T)$ is deterministic with the form
\begin{eqnarray}
    \sigma^f(t,T) = \sigma_{t} e^{-\int_{t}^T\lambda _{s}ds}
    \label{SF_def}
\end{eqnarray}

Based on the relation between $\sigma^B(t,T)$ and $\sigma^f(t,T)$ under HJM model, one can deduce that
\begin{eqnarray}
    \sigma^B(t,T) = -\int_t^T \sigma^f(t,s) ds = -\sigma_{t} \Lambda (t,T)  \label{SigmaB_def}
\end{eqnarray}
with
\[
    \Lambda (t,T)=\int_{t}^{T}e^{-\int_{t}^{s}\lambda _{u}du}ds
    \label{lamda_def}    
\]

$\lambda_{t}$ can be seen the mean reversion rate of the short rate and is
also assumed to be deterministic. It is the dampening factor for the
volatility of the forward rate. If $\lambda$ is constant, then $\Lambda_{i}
(t,T)= \frac{1-e^{-\lambda(T-t)}}{\lambda}$.

In practice usually a multi-factor LGM model is used. In following sections by default we are in 
a three factor LGM world, that is, $f(t,T)$ is driven by a correlated three dimensional Brownian motion
$W$.

By introducing the forward measure $Q^T$ under which the forward rate is a
martingale, one writes the dynamics of the forward rate as 
\[
    f(t,T)=\sigma^f(t,T)\cdot dW^T_{t}    
\]

By definition in equation (\ref{r_t_def}) the spot rate under 3-factor model
is expressed as
\begin{eqnarray}
r_{t} = f(t,t)=f(0,t)+ \sum^{3}_{i=1}x_{t}^i  \label{r_int}
\end{eqnarray}
\begin{eqnarray}
    x^i_{t} = \int_{0}^t \sigma_{i}^f(s,t)dW_{s}^{t,i} =
    -\int_{0}^t\sum_{j=1}^3\rho_{ij}\sigma_{j}^B(s,t)\sigma^f_{i}(s,t)ds+
    \sigma^f_{i}(s,t)dW_{s}^{i}  
    \label{x_int}
\end{eqnarray}
where $W_{s}^{i}$ is the Brownian motion under the risk neutral measure $Q$.

It is noted that the deterministic property of $\sigma_{t}^i$ and $
\lambda_{t}$ functions assures that the spot rate is Markovian. Indeed from
equation and (\ref{r_int}) and (\ref{x_int}), $r_{t}$ is a function of $t$
and $x_{t}$, where $x_{t}$ is a sum of a deterministic function and a
Gaussian internal. Gaussian process are Markovian so is $r_{t}$.


\begin{prop}
    The factor $x_{t}^i$ evovles as 
    \begin{eqnarray}
        dx^i_{t} = \left(\sum_{j}\Phi_{t}^{ij}-\lambda_{t}^ix_{t}^i \right)dt+
        \sigma_{t}^i dW_{t}^{i}  \label{x_SDE}
    \end{eqnarray}
    where $\Phi^{ij}_{t}= \mathrm{Cov}(x_{t}^i, x_{t}^j) =
    \int_{0}^t\rho_{ij}\sigma^f_{j}(s,t)\sigma_{i}^f(s,t)ds$ is the covariance
    matrix of the factors.
\end{prop}

\begin{proof}
    By equation (\ref{x_int}) $x_{t}$ has the dynamics
\begin{eqnarray*}
    dx_{t}^{i}&=&-\int_{0}^{t}\underset{j}{\sum }\rho _{ij}d(\sigma
    _{j}^{B}(s,t)\sigma _{i}^{f}(s,t))ds-\underset{j}{\sum }\rho _{ij}\sigma
    _{j}^{B}(t,t)\sigma _{i}^{f}(t,t)+\sigma _{i}^{f}(t,t)dW_{s}^{\ast ,i} \\
            &+&\int_{0}^{t}d\sigma _{i}^{f}(s,t)dW_{s}^{i}
\end{eqnarray*}

From equation (\ref{SF_def}) and (\ref{SigmaB_def}), it can be verified that
\begin{eqnarray*}
    d(\sigma _{j}^{B}(s,t)\sigma _{i}^{f}(s,t)) &=&\sigma _{j}^{B}(s,t)d\sigma
    _{i}^{f}(s,t)+\sigma _{i}^{f}(s,t)d\sigma _{j}^{B}(s,t) \\
    &=& -\lambda _{t}^{i}\sigma _{i}^{f}(s,t)\sigma _{j}^{B}(s,t)dt-\sigma
    _{i}^{f}(s,t)\sigma _{j}^{f}(s,t)dt
\end{eqnarray*}

As a result, it is obtained that
\[
    \int_{0}^{t}\underset{j}{\sum }\rho _{ij}d(\sigma _{j}^{B}(s,t)\sigma
    _{i}^{f}(s,t))ds = \int_{0}^{t}\underset{j}{\sum }\rho _{ij}\lambda
    _{t}^{i}\sigma _{j}^{B}(s,t)\sigma _{i}^{f}(s,t)dtds-\int_{0}^{t}\underset{j}
    {\sum }\rho _{ij}\sigma _{i}^{f}(s,t)\sigma _{j}^{f}(s,t)dtds
\]

By definition of $\sigma^{B}(t,T)$ in equation (\ref{SigmaB_def}), one
obtains
\[
    \sigma _{j}^{B}(t,t)=0
\]

By using Fubini's theorem for stochastic integral, one can evaluate the
following integral as
\[
    \int_{0}^{t}d\sigma _{i}^{f}(s,t)dW_{s}^{i}=\int_{0}^{t}-\lambda
    _{t}^{i}\sigma _{i}^{f}(s,t)dtdW_{s}^{i}=\int_{0}^{t}-\lambda _{t}^{i}\sigma
    _{i}^{f}(s,t)dW_{s}^{i}dt
\]

Therefore the dynamics of $x_{t}$ is equivalent to
\begin{eqnarray*}
    dx_{t}^{i}&=&-\int_{0}^{t}\underset{j}{\sum }\rho _{ij}\lambda
    _{t}^{i}\sigma _{j}^{B}(s,t)\sigma _{i}^{f}(s,t)dtds+\int_{0}^{t}\underset{j}
    {\sum }\rho _{ij}\sigma _{i}^{f}(s,t)\sigma _{j}^{f}(s,t)dtds+\sigma
    _{j}^{f}(t,t)dW_{s}^{\ast ,i} \\
    &-&\int_{0}^{t}\lambda _{t}^{i}\sigma _{i}^{f}(s,t)dW_{s}^{\ast ,i}dt
\end{eqnarray*}

By switching the integral order again, it is further equivalent to
\begin{eqnarray*}
    dx_{t}^{i}&=& \prt{\int_{0}^{t}\underset{j}{\sum }\rho _{ij}\sigma
    _{i}^{f}(s,t)\sigma _{j}^{f}(s,t)ds-\lambda _{t}^{i}x_{t}^{i}}dt+\sigma
    _{t}^{i}dW_{s}^{i} \\
    &=&(\underset{j}{\sum }\Phi _{t}^{ij}-\lambda _{t}^{i}x_{t}^{i})dt+\sigma
    _{t}^{i}dW_{s}^{i}
\end{eqnarray*}
where
\[
    \Phi _{t}^{ij}=cov(x_{t}^{i},x_{t}^{j})=\int_{0}^{t}\rho _{ij}\sigma
    _{i}^{f}(s,t)\sigma _{j}^{f}(s,t)ds   
\]
\end{proof}

\newpage

\begin{thebibliography}{99}
\bibitem{Shreve} Steven Shreve, Stochastic Calculus for Finance II, {\it Springer Finance},  Springer, 2004. 

\bibitem{Mercurio} Damiano Brigo and Fabio Mercurio. Interest rate models: theory and practice. {\it Springer Finance, 2nd edition}, Springer, 2006.

\bibitem{Lewis} Harvey Stein. Valuation of exotic interest rate derivatives. {\it research note}, Bloomberg LP, December, 2007.

\bibitem{Zeng} Yan Zeng. Elements of LGM model. {\it
work note, version 1.0}, July, 2012.

\bibitem{Xiong} Changwei Xiong. Introduction to interest rate models, {\it work note}, Nomura Singapore, June, 2018.

\end{thebibliography}

\end{document} 